\documentclass{article}
\usepackage{amsmath}
\usepackage{amsfonts}
\usepackage[a4paper,width=140mm,top=15mm,bottom=15mm]{geometry}
\usepackage{hyperref}
\usepackage{mathtools}
\usepackage{graphicx}
\graphicspath{ {./figures/} }

\hypersetup{
    colorlinks,
    citecolor=black,
    filecolor=black,
    linkcolor=black,
    urlcolor=black
}

\DeclareUnicodeCharacter{2212}{-}




\title{Esercizi}
\author{Michele Leigheb}
\date{}
\begin{document}
\maketitle
\tableofcontents{}
\section{Complessi}
\begin{itemize}
	\item \(\displaystyle 2^{(a+ib)} = 2^a (\cos(b \ln(2)) + i\sin(b \ln(2))) \)
	\item \(\displaystyle 3^{(a+ib)} = 3^a (\cos(b \ln(3)) + i\sin(b \ln(3))) \)
	\item \(\displaystyle e^{(a+ib)} = e^a (\cos(b)) + i\sin(b) \)
	\item \(\displaystyle \alpha^{(a+ib)} = e^{\alpha} (\cos(b)\ln(\alpha)) + i\sin(b\ln(\alpha)) \)
\end{itemize}



\section{Esercizi}

\[ W(s) = \frac{\left(\begin{matrix}s & 0 & 1\\1 & s + 1 & 0\end{matrix}\right)}{s^{2} + 1} \]
\subsubsection{Forma Canonica Raggiungibile}
Con la realizzazione cano ragg viene \[ A = \left(\begin{matrix}0 & 0 & 0 & 1 & 0 & 0\\0 & 0 & 0 & 0 & 1 & 0\\0 & 0 & 0 & 0 & 0 & 1\\-1 & 0 & 0 & 0 & 0 & 0\\0 & -1 & 0 & 0 & 0 & 0\\0 & 0 & -1 & 0 & 0 & 0\end{matrix}\right), B = \left(\begin{matrix}0 & 0 & 0\\0 & 0 & 0\\0 & 0 & 0\\1 & 0 & 0\\0 & 1 & 0\\0 & 0 & 1\end{matrix}\right), C= \left(\begin{matrix}0 & 0 & 1 & 1 & 0 & 0\\1 & 1 & 0 & 0 & 1 & 0\end{matrix}\right) \]




























\end{document}
