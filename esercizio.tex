\documentclass{article}
\usepackage{amsmath}
\usepackage{amsfonts}
\usepackage[a4paper,width=140mm,top=15mm,bottom=15mm]{geometry}
\usepackage{hyperref}
\usepackage{mathtools}
\usepackage{graphicx}
\graphicspath{ {./figures/} }

\hypersetup{
    colorlinks,
    citecolor=black,
    filecolor=black,
    linkcolor=black,
    urlcolor=black
}

\DeclareUnicodeCharacter{2212}{-}




\title{Esercizi}
\author{Michele Leigheb}
\date{}
\begin{document}
\maketitle
\tableofcontents{}
\section{Complessi}
\begin{itemize}
	\item \(\displaystyle 2^{(a+ib)} = 2^a (\cos(b \ln(2)) + i\sin(b \ln(2))) \)
	\item \(\displaystyle 3^{(a+ib)} = 3^a (\cos(b \ln(3)) + i\sin(b \ln(3))) \)
	\item \(\displaystyle e^{(a+ib)} = e^a (\cos(b)) + i\sin(b) \)
	\item \(\displaystyle \alpha^{(a+ib)} = e^{\alpha} (\cos(b)\ln(\alpha)) + i\sin(b\ln(\alpha)) \)
\end{itemize}



\section{Esercizi}

\section{Esercizio 158 }
 Studiare il sistema \[S:\begin{cases}\overset{\cdot}{x} = \left(\begin{matrix}0 & 1 & 1\\0 & -1 & 0\\0 & 0 & -1\end{matrix}\right) x+ \left(\begin{matrix}1\\0\\0\end{matrix}\right)u\\y = \left(\begin{matrix}1 & 0 & 1\end{matrix}\right) x +\left(\begin{matrix}0\end{matrix}\right) u\end{cases}\]\subsection{Studio Risposta Libera}
Si studi la risposta libera di un sistema che ha le seguenti caratteristiche: \[A = \left(\begin{matrix}0 & 1 & 1\\0 & -1 & 0\\0 & 0 & -1\end{matrix}\right)\]
Il determinante di $A-\lambda$ è $ - \lambda^{3} - 2 \lambda^{2} - \lambda $.

Gli autovalori reali sono $\lambda_i = [-1, 0]$.

Gli autovettori associati ai reali sono $ u_i: [  -1: (\left(\begin{matrix}-1\\1\\0\end{matrix}\right), \left(\begin{matrix}-1\\0\\1\end{matrix}\right))0: \left(\begin{matrix}1\\0\\0\end{matrix}\right) ]$
.

Da cui posso ricavare le matrici \[U=T^{-1} = \left(\begin{matrix}-1 & -1 & 1\\1 & 0 & 0\\0 & 1 & 0\end{matrix}\right), V = T = \left(\begin{matrix}0 & 1 & 0\\0 & 0 & 1\\1 & 1 & 1\end{matrix}\right)\]
Che mi trasformano la matrice in \[ D = TAT^{-1} = \left(\begin{matrix}-1 & 0 & 0\\0 & -1 & 0\\0 & 0 & 0\end{matrix}\right) \]
Da cui posso ricavare: \[ \Phi(t) = e^{At} = T^{-1} e^{Dt} T =  T^{-1} \left(\begin{matrix}e^{- t} & 0 & 0\\0 & e^{- t} & 0\\0 & 0 & 1\end{matrix}\right) T\]

\[ = \left(\begin{matrix}1 & 1 - e^{- t} & 1 - e^{- t}\\0 & e^{- t} & 0\\0 & 0 & e^{- t}\end{matrix}\right) \]\[ \Psi(t) = \left(\begin{matrix}1 & 1 - e^{- t} & 1\end{matrix}\right), H(t) =  \left(\begin{matrix}1\\0\\0\end{matrix}\right),W(t) = \left(\begin{matrix}1\end{matrix}\right)\]\subsubsection{Osservabilità}
 I modi naturali osservabili sono quelli tali che 
\[ C \cdot u_i   \neq 0\]
\subsubsection{Eccitabilità}
 I modi naturali eccitabili sono quelli tali che 
\[v_i' \cdot B \neq 0\]

\subsection{Studio Osservabilità}

Studiamone l'osservabilità. Calcoliamo allora $O$ e troviamo $\mathfrak{I} = \text{ker}(O)$:
\[
 O = \begin{pmatrix}C \\ ... \\ CA^{n-1}  \end{pmatrix} = \left(\begin{matrix}1 & 0 & 1\\0 & 1 & 0\\0 & -1 & 0\end{matrix}\right), |O| = 0 \]

O ha rango $ 2 $ quindi il suo nucleo ha dimensione $ 1 $.

Calcolando trovo \[ 
I = ker(O) = \left[ \left(\begin{matrix}-1\\0\\1\end{matrix}\right)\right]\]

E $T^{-1}$ viene \[ 
T^{-1} = \left(\begin{matrix}-1 & 0 & 0\\0 & 1 & 0\\1 & 0 & 1\end{matrix}\right) \]
\[ 
T = \left(\begin{matrix}-1 & 0 & 0\\0 & 1 & 0\\1 & 0 & 1\end{matrix}\right) \]Calcoliamo allora le matrici del sistema, e vedremo che risultano partizionate come avevamo previsto:
\[ 
\overset{\sim}{A} = T A  T^{-1} = \left(\begin{matrix}-1 & 0 & 0\\0 & 1 & 0\\1 & 0 & 1\end{matrix}\right)\left(\begin{matrix}0 & 1 & 1\\0 & -1 & 0\\0 & 0 & -1\end{matrix}\right)\left(\begin{matrix}-1 & 0 & 0\\0 & 1 & 0\\1 & 0 & 1\end{matrix}\right) = \left(\begin{matrix}-1 & -1 & -1\\0 & -1 & 0\\0 & 1 & 0\end{matrix}\right) \]
\[ 
\overset{\sim}{C} = CT^{-1} = \left(\begin{matrix}1 & 0 & 1\end{matrix}\right)\left(\begin{matrix}-1 & 0 & 0\\0 & 1 & 0\\1 & 0 & 1\end{matrix}\right) = \left(\begin{matrix}0 & 0 & 1\end{matrix}\right) = ( 0\ \ \overset{\sim}{C}_2) \]
Con le matrici \[ A_{11} = \left(\begin{matrix}-1\end{matrix}\right) , A_{22} = \left(\begin{matrix}-1 & 0\\1 & 0\end{matrix}\right) \]Quindi infine mi viene che gli autovalori osservabili sono $ -1\ 0\  $ e gli inosservabili sono $ -1\  $.

\subsection{Studio Raggiungibilità}
Ora per studiare la raggiungibilità degli stati calcolo $R = (B\ AB\ ...\ A^{n-1}B)$: \[ R = \left(\begin{matrix}1 & 0 & 0\\0 & 0 & 0\\0 & 0 & 0\end{matrix}\right), |R| = 0 \] 

Vediamo che $rango(R) = 1$ e quindi : \[ \mathfrak{R} = \left[ \left(\begin{matrix}1\\0\\0\end{matrix}\right)\right] \]

Tutti gli stati che hanno questa struttura sono, allora, raggiungibili. Mettiamo in evidenza questa struttura;
cambiamo base, e vorremmo avere lo stato espresso come $z = Tx$ tale che, se x è uno stato raggiungibile, allora: \[ z_R = T x_R = \begin{pmatrix} \star  \\ 0 \\0\end{pmatrix}\]

E $T^{-1}$ viene \[ T^{-1} = \left(\begin{matrix}1 & 0 & 0\\0 & 1 & 0\\0 & 0 & 1\end{matrix}\right) \Longrightarrow T = \left(\begin{matrix}1 & 0 & 0\\0 & 1 & 0\\0 & 0 & 1\end{matrix}\right) \]
\[ \overset{\sim}{A} = T A  T^{-1} = \left(\begin{matrix}1 & 0 & 0\\0 & 1 & 0\\0 & 0 & 1\end{matrix}\right)\left(\begin{matrix}0 & 1 & 1\\0 & -1 & 0\\0 & 0 & -1\end{matrix}\right)\left(\begin{matrix}1 & 0 & 0\\0 & 1 & 0\\0 & 0 & 1\end{matrix}\right) = \left(\begin{matrix}0 & 1 & 1\\0 & -1 & 0\\0 & 0 & -1\end{matrix}\right) \]Con le matrici \[ \overset{\sim}{A}_{11} = \left(\begin{matrix}0\end{matrix}\right) , \overset{\sim}{A}_{22} = \left(\begin{matrix}-1 & 0\\0 & -1\end{matrix}\right)  \]e le matrici \[ \overset{\sim}{B} = TB = \left(\begin{matrix}1\\0\\0\end{matrix}\right)  \]
Ora ne calcoliamo la raggiungibilità: \[ \overset{\sim}{H}(t) = e^{\overset{\sim}{A}t}\overset{\sim}{B} = \begin{pmatrix} e^{\overset{\sim}{A}_{11}t} &  \star \\ 0 & e^{\overset{\sim}{A}_{22}t} \end{pmatrix} \begin{pmatrix} \overset{\sim}{B}_1 \\ 0 \end{pmatrix} = \begin{pmatrix} e^{\overset{\sim}{A_{11}t}}\overset{\sim}{B_1} \\ 0 \end{pmatrix} \]
Quindi infine mi viene che gli autovalori ragg sono $ 0,  $ e gli irrag sono $ -1,  $
\subsection{Scomposizione di Kalman}
I miei sottospazi di riferimento sono:	\[ \mathfrak{I} = \left[ \left(\begin{matrix}-1\\0\\1\end{matrix}\right)\right], \mathfrak{R} = \left[ \left(\begin{matrix}1\\0\\0\end{matrix}\right)\right] \]
La matrice dei vettori di base di I e R è \[ \chi_1 =  \left(\begin{matrix}\end{matrix}\right) \]Ed il determinante è ininfluente.
\paragraph{Per quanto riguarda Chi2:} $ \chi_2 | \chi_2 \oplus \chi_1 = \mathfrak{R} $ è \[ \chi_2 = \left(\begin{matrix}1\\0\\0\end{matrix}\right) \]

\paragraph{Per quanto riguarda Chi3:} $ \chi_3 | \chi_3 \oplus \chi_1 = \mathfrak{I} $ è \[ \chi_3 = \left(\begin{matrix}-1\\0\\1\end{matrix}\right) \]

\paragraph{Per quanto riguarda Chi4:} $ \chi_4 | \chi_1 \oplus \chi_2 \oplus  \chi_3 \oplus \chi_4 = \mathbb{R} $ è \[ \chi_4 = \left(\begin{matrix}0\\1\\0\end{matrix}\right) \]
\paragraph{Ora facciamo T inversa:} \[ T^{-1} = (\chi_1\ \chi_2\ \chi_3\ \chi_4\ ) = \left(\begin{matrix}1 & -1 & 0\\0 & 0 & 1\\0 & 1 & 0\end{matrix}\right) \]
e quindi \[T = \left(\begin{matrix}1 & 0 & 1\\0 & 0 & 1\\0 & 1 & 0\end{matrix}\right)\]
\[ \widetilde{A} = TAT^{-1} = \left(\begin{matrix}0 & 1 & 0\\0 & 0 & -1\\0 & -1 & 0\end{matrix}\right) * T^{-1} = T*\left(\begin{matrix}0 & 1 & 1\\0 & 0 & -1\\0 & -1 & 0\end{matrix}\right) =\left(\begin{matrix}0 & 0 & 1\\0 & -1 & 0\\0 & 0 & -1\end{matrix}\right) \]

\[ \widetilde{B} = T B = \left(\begin{matrix}1\\0\\0\end{matrix}\right) \]

\[ \widetilde{C} = C T^{-1} = \left(\begin{matrix}1 & 0 & 0\end{matrix}\right) \]
\[Phi(t) = \left(\begin{matrix}1 & 0 & 1 - e^{- t}\\0 & e^{- t} & 0\\0 & 0 & e^{- t}\end{matrix}\right) \]

\subsection{Studio Funzione di trasferimento}

\[ (sI-A) = \left(\begin{matrix}s & -1 & -1\\0 & s + 1 & 0\\0 & 0 & s + 1\end{matrix}\right), |sI-A| = s \left(s + 1\right)^{2} \]
\[ \Phi(s) = (sI-A)^{-1} = \frac{\left(\begin{matrix}\left(s + 1\right)^{2} & s + 1 & s + 1\\0 & s \left(s + 1\right) & 0\\0 & 0 & s \left(s + 1\right)\end{matrix}\right)}{s \left(s + 1\right)^{2}} \]

Le funzioni caratteristiche sono \[\begin{array}{rcl}  H(s) & = & \Phi(s)B \\ \Psi(s) & = & C \Phi(s)\\ W(s) & = & C(sI-A)^{-1}B  \end{array} \]
\[ H(s) = \left(\begin{matrix}\left(s + 1\right)^{2} & s + 1 & s + 1\\0 & s \left(s + 1\right) & 0\\0 & 0 & s \left(s + 1\right)\end{matrix}\right)\cdot \frac{\left(\begin{matrix}1\\0\\0\end{matrix}\right)}{s \left(s + 1\right)^{2}} = \frac{\left(\begin{matrix}\left(s + 1\right)^{2}\\0\\0\end{matrix}\right)}{s \left(s + 1\right)^{2}} \]\[ \Psi(s) = \frac{\left(\begin{matrix}1 & 0 & 1\end{matrix}\right)}{s \left(s + 1\right)^{2}}\cdot\left(\begin{matrix}\left(s + 1\right)^{2} & s + 1 & s + 1\\0 & s \left(s + 1\right) & 0\\0 & 0 & s \left(s + 1\right)\end{matrix}\right)  = \frac{\left(\begin{matrix}\left(s + 1\right)^{2} & s + 1 & s \left(s + 1\right) + s + 1\end{matrix}\right)}{s \left(s + 1\right)^{2}} \]
e quindi \[ H(s)  =  \frac{\left(\begin{matrix}\left(s + 1\right)^{2}\\0\\0\end{matrix}\right)}{s \left(s + 1\right)^{2}} \ \Psi(s) = \frac{\left(\begin{matrix}\left(s + 1\right)^{2} & s + 1 & s \left(s + 1\right) + s + 1\end{matrix}\right)}{s \left(s + 1\right)^{2}} \]
\[ W(s)  =  \frac{\left(\begin{matrix}\left(s + 1\right)^{2}\end{matrix}\right)}{s \left(s + 1\right)^{2}} = \left(\begin{matrix}\frac{1}{s}\end{matrix}\right)  \] 
Il grafico di bode è:
\[ W(s) = \frac{\left(s + 1\right)^{2}}{s \left(s + 1\right)^{2}} \]\includegraphics[scale = 0.5]{figures/bode_4164632.png}


\subsubsection{Vediamo le risposte:} 




























\end{document}
