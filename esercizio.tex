\documentclass{article}
\usepackage{amsmath}
\usepackage{amsfonts}
\usepackage[a4paper,width=140mm,top=15mm,bottom=15mm]{geometry}
\usepackage{hyperref}
\usepackage{mathtools}
\usepackage{graphicx}
\graphicspath{ {./figures/} }

\hypersetup{
    colorlinks,
    citecolor=black,
    filecolor=black,
    linkcolor=black,
    urlcolor=black
}

\DeclareUnicodeCharacter{2212}{-}




\title{Esercizi}
\author{Michele Leigheb}
\date{}
\begin{document}
\maketitle
\tableofcontents{}
\section{Complessi}
\begin{itemize}
	\item \(\displaystyle 2^{(a+ib)} = 2^a (\cos(b \ln(2)) + i\sin(b \ln(2))) \)
	\item \(\displaystyle 3^{(a+ib)} = 3^a (\cos(b \ln(3)) + i\sin(b \ln(3))) \)
	\item \(\displaystyle e^{(a+ib)} = e^a (\cos(b)) + i\sin(b) \)
	\item \(\displaystyle \alpha^{(a+ib)} = e^{\alpha} (\cos(b)\ln(\alpha)) + i\sin(b\ln(\alpha)) \)
\end{itemize}



\section{Esercizi}

\section{Esercizio 77 }
 Studiare il sistema \[S:\begin{cases}\overset{\cdot}{x} = \left(\begin{matrix}0 & 1 & 0 & 0\\0 & -1 & 0 & 1\\0 & 0 & 0 & 0\\0 & 0 & 1 & -2\end{matrix}\right) x+ \left(\begin{matrix}0 & 0\\0 & 0\\2 & 0\\1 & 1\end{matrix}\right)u\\y = \left(\begin{matrix}1 & 1 & 0 & 0\\2 & 0 & 0 & 0\end{matrix}\right) x +\left(\begin{matrix}0 & 0\\0 & 0\end{matrix}\right) u\end{cases}\]\subsection{Studio Risposta Libera}
Si studi la risposta libera di un sistema che ha le seguenti caratteristiche: \[A = \left(\begin{matrix}0 & 1 & 0 & 0\\0 & -1 & 0 & 1\\0 & 0 & 0 & 0\\0 & 0 & 1 & -2\end{matrix}\right)\]
Il determinante di $A-\lambda$ è $ \lambda^{2} \left(- \lambda - 2\right) \left(- \lambda - 1\right) $.

Gli autovalori reali sono $\lambda_i = [-2, -1, 0]$.

Gli autovettori associati ai reali sono $ u_i: [  -2: \left(\begin{matrix}\frac{1}{2}\\-1\\0\\1\end{matrix}\right)-1: \left(\begin{matrix}-1\\1\\0\\0\end{matrix}\right)0: (\left(\begin{matrix}1\\0\\0\\0\end{matrix}\right)) ]$
.
La matrice $A$ non è diagonalizzabile, quindi devo fare Jordan.

Da cui posso ricavare le matrici \[U=T^{-1} = \left(\begin{matrix}\frac{1}{2} & -1 & 1 & 0\\-1 & 1 & 0 & 1\\0 & 0 & 0 & 2\\1 & 0 & 0 & 1\end{matrix}\right), V = T = \left(\begin{matrix}0 & 0 & - \frac{1}{2} & 1\\0 & 1 & -1 & 1\\1 & 1 & - \frac{3}{4} & \frac{1}{2}\\0 & 0 & \frac{1}{2} & 0\end{matrix}\right)\]
Che mi trasformano la matrice in \[ D = TAT^{-1} = \left(\begin{matrix}-2 & 0 & 0 & 0\\0 & -1 & 0 & 0\\0 & 0 & 0 & 1\\0 & 0 & 0 & 0\end{matrix}\right) \]
Da cui posso ricavare: \[ \Phi(t) = e^{At} = T^{-1} e^{Dt} T =  T^{-1} \left(\begin{matrix}e^{- 2 t} & 0 & 0 & 0\\0 & e^{- t} & 0 & 0\\0 & 0 & 1 & t\\0 & 0 & 0 & 1\end{matrix}\right) T\]

\[ = \left(\begin{matrix}1 & 1 - e^{- t} & \frac{t}{2} - \frac{3}{4} + e^{- t} - \frac{e^{- 2 t}}{4} & \frac{1}{2} - e^{- t} + \frac{e^{- 2 t}}{2}\\0 & e^{- t} & \frac{1}{2} - e^{- t} + \frac{e^{- 2 t}}{2} & e^{- t} - e^{- 2 t}\\0 & 0 & 1 & 0\\0 & 0 & \frac{1}{2} - \frac{e^{- 2 t}}{2} & e^{- 2 t}\end{matrix}\right) \]\subsubsection{Osservabilità}
 I modi naturali osservabili sono quelli tali che 
\[ C \cdot u_i   \neq 0\]
\subsubsection{Eccitabilità}
 I modi naturali eccitabili sono quelli tali che 
\[v_i' \cdot B \neq 0\]

\subsection{Studio Osservabilità}

Studiamone l’osservabilità. Calcoliamo allora O e troviamo I = ker(O):
\[
 O = \begin{pmatrix}C \\ CA \\ CA^2 \end{pmatrix} = \left(\begin{matrix}1 & 1 & 0 & 0\\2 & 0 & 0 & 0\\0 & 0 & 0 & 1\\0 & 2 & 0 & 0\\0 & 0 & 1 & -2\\0 & -2 & 0 & 2\\0 & 0 & -2 & 4\\0 & 2 & 2 & -6\end{matrix}\right), |O| = 999 \]

O ha rango $ 4 $ quindi il suo nucleo ha dimensione $ 4 $.

Calcolando trovo \[ 
I = ker(O) = \left[ \right]\]
2O ha rango pieno quindi finisco qui.

\subsection{Studio Raggiungibilità}
Ora per studiare la raggiungibilità degli stati calcolo $R = (B\ AB\ ...\ A^{n-1}B)$: \[ R = \left(\begin{matrix}0 & 0 & 0 & 0 & 1 & 1 & -1 & -3\\0 & 0 & 1 & 1 & -1 & -3 & 1 & 7\\2 & 0 & 0 & 0 & 0 & 0 & 0 & 0\\1 & 1 & 0 & -2 & 0 & 4 & 0 & -8\end{matrix}\right), |R| = 999 \] 

Vediamo che $rango(R) = 4$ e quindi : \[ \mathfrak{R} = \left[ \left(\begin{matrix}0\\0\\2\\1\end{matrix}\right), \  \left(\begin{matrix}0\\0\\0\\1\end{matrix}\right), \  \left(\begin{matrix}0\\1\\0\\0\end{matrix}\right), \  \left(\begin{matrix}1\\-1\\0\\0\end{matrix}\right)\right] \]

Tutti gli stati che hanno questa struttura sono, allora, raggiungibili. Mettiamo in evidenza questa struttura;
cambiamo base, e vorremmo avere lo stato espresso come $z = Tx$ tale che, se x è uno stato raggiungibile, allora: \[ z_R = T x_R = \begin{pmatrix} \star  \\ 0 \\0\end{pmatrix}\]

E $T^{-1}$ viene \[ T^{-1} = \left(\begin{matrix}0 & 0 & 0 & 1\\0 & 0 & 1 & -1\\2 & 0 & 0 & 0\\1 & 1 & 0 & 0\end{matrix}\right) \Longrightarrow T = \left(\begin{matrix}0 & 0 & \frac{1}{2} & 0\\0 & 0 & - \frac{1}{2} & 1\\1 & 1 & 0 & 0\\1 & 0 & 0 & 0\end{matrix}\right) \]
\[ \overset{\sim}{A} = T A  T^{-1} = \left(\begin{matrix}0 & 0 & \frac{1}{2} & 0\\0 & 0 & - \frac{1}{2} & 1\\1 & 1 & 0 & 0\\1 & 0 & 0 & 0\end{matrix}\right)\left(\begin{matrix}0 & 1 & 0 & 0\\0 & -1 & 0 & 1\\0 & 0 & 0 & 0\\0 & 0 & 1 & -2\end{matrix}\right)\left(\begin{matrix}0 & 0 & 0 & 1\\0 & 0 & 1 & -1\\2 & 0 & 0 & 0\\1 & 1 & 0 & 0\end{matrix}\right) = \left(\begin{matrix}0 & 0 & 0 & 0\\0 & -2 & 0 & 0\\1 & 1 & 0 & 0\\0 & 0 & 1 & -1\end{matrix}\right) \]Con le matrici \[ \overset{\sim}{A}_{11} = \left(\begin{matrix}0 & 0 & 0 & 0\\0 & -2 & 0 & 0\\1 & 1 & 0 & 0\\0 & 0 & 1 & -1\end{matrix}\right) , \overset{\sim}{A}_{22} = \left(\begin{matrix}\end{matrix}\right)  \]e le matrici \[ \overset{\sim}{B} = TB = \left(\begin{matrix}1 & 0\\0 & 1\\0 & 0\\0 & 0\end{matrix}\right)  \]
Ora ne calcoliamo la raggiungibilità: \[ \overset{\sim}{H}(t) = e^{\overset{\sim}{A}t}\overset{\sim}{B} = \begin{pmatrix} e^{\overset{\sim}{A}_{11}t} &  \star \\ 0 & e^{\overset{\sim}{A}_{22}t} \end{pmatrix} \begin{pmatrix} \overset{\sim}{B}_1 \\ 0 \end{pmatrix} = \begin{pmatrix} e^{\overset{\sim}{A_{11}t}}\overset{\sim}{B_1} \\ 0 \end{pmatrix} \]
Quindi infine mi viene che gli autovalori ragg sono $ 0, -2, -1,  $ e gli irrag sono $  $
\subsection{Scomposizione di Kalman}
I miei sottospazi di riferimento sono:	\[ \mathfrak{I} = \left[ \right], \mathfrak{R} = \left[ \left(\begin{matrix}0\\0\\2\\1\end{matrix}\right), \  \left(\begin{matrix}0\\0\\0\\1\end{matrix}\right), \  \left(\begin{matrix}0\\1\\0\\0\end{matrix}\right), \  \left(\begin{matrix}1\\-1\\0\\0\end{matrix}\right)\right] \]
La matrice dei vettori di base di I e R è \[ \chi_1 =  \left(\begin{matrix}\end{matrix}\right) \]Ed il determinante è ininfluente.
\paragraph{Per quanto riguarda Chi2:} $ \chi_2 | \chi_2 \oplus \chi_1 = \mathfrak{R} $ è \[ \chi_2 = \left(\begin{matrix}0 & 0 & 0 & 1\\0 & 0 & 1 & -1\\2 & 0 & 0 & 0\\1 & 1 & 0 & 0\end{matrix}\right) \]

\paragraph{Per quanto riguarda Chi3:} $ \chi_3 | \chi_3 \oplus \chi_1 = \mathfrak{I} $ è \[ \chi_3 = \left(\begin{matrix}\end{matrix}\right) \]

\paragraph{Per quanto riguarda Chi4:} $ \chi_4 | \chi_1 \oplus \chi_2 \oplus  \chi_3 \oplus \chi_4 = \mathbb{R} $ è \[ \chi_4 = \left(\begin{matrix}\end{matrix}\right) \]
\paragraph{Ora facciamo T inversa:} \[ T^{-1} = (\chi_1\ \chi_2\ \chi_3\ \chi_4\ ) = \left(\begin{matrix}0 & 0 & 0 & 1\\0 & 0 & 1 & -1\\2 & 0 & 0 & 0\\1 & 1 & 0 & 0\end{matrix}\right) \]
e quindi \[T = \left(\begin{matrix}0 & 0 & \frac{1}{2} & 0\\0 & 0 & - \frac{1}{2} & 1\\1 & 1 & 0 & 0\\1 & 0 & 0 & 0\end{matrix}\right)\]
\[ \widetilde{A} = TAT^{-1} = \left(\begin{matrix}0 & 0 & 0 & 0\\0 & 0 & 1 & -2\\0 & 0 & 0 & 1\\0 & 1 & 0 & 0\end{matrix}\right) * T^{-1} = T*\left(\begin{matrix}0 & 0 & 1 & -1\\1 & 1 & -1 & 1\\0 & 0 & 0 & 0\\0 & -2 & 0 & 0\end{matrix}\right) =\left(\begin{matrix}0 & 0 & 0 & 0\\0 & -2 & 0 & 0\\1 & 1 & 0 & 0\\0 & 0 & 1 & -1\end{matrix}\right) \]

\[ \widetilde{B} = T B = \left(\begin{matrix}1 & 0\\0 & 1\\0 & 0\\0 & 0\end{matrix}\right) \]

\[ \widetilde{C} = C T^{-1} = \left(\begin{matrix}0 & 0 & 1 & 0\\0 & 0 & 0 & 2\end{matrix}\right) \]
\[Phi(t) = \left(\begin{matrix}1 & 0 & 0 & 0\\0 & e^{- 2 t} & 0 & 0\\t & \frac{1}{2} - \frac{e^{- 2 t}}{2} & 1 & 0\\t - 1 + e^{- t} & \frac{1}{2} - e^{- t} + \frac{e^{- 2 t}}{2} & 1 - e^{- t} & e^{- t}\end{matrix}\right) \]

\subsection{Studio Funzione di trasferimento}

\[ (sI-A) = \left(\begin{matrix}s & -1 & 0 & 0\\0 & s + 1 & 0 & -1\\0 & 0 & s & 0\\0 & 0 & -1 & s + 2\end{matrix}\right), |sI-A| = s^{2} \left(s + 1\right) \left(s + 2\right) \]
\[ \Phi(s) = (sI-A)^{-1} = \frac{\left(\begin{matrix}s \left(s + 1\right) \left(s + 2\right) & s \left(s + 2\right) & 1 & s\\0 & s^{2} \left(s + 2\right) & s & s^{2}\\0 & 0 & s \left(s + 1\right) \left(s + 2\right) & 0\\0 & 0 & s \left(s + 1\right) & s^{2} \left(s + 1\right)\end{matrix}\right)}{s^{2} \left(s + 1\right) \left(s + 2\right)} \]

Le funzioni caratteristiche sono \[\begin{array}{rcl}  H(s) & = & \Phi(s)B \\ \Psi(s) & = & C \Phi(s)\\ W(s) & = & C(sI-A)^{-1}B  \end{array} \]

e quindi \[ H(s)  =  \frac{\left(\begin{matrix}s + 2 & s\\s \left(s + 2\right) & s^{2}\\2 s \left(s + 1\right) \left(s + 2\right) & 0\\s \left(s + 1\right) \left(s + 2\right) & s^{2} \left(s + 1\right)\end{matrix}\right)}{s^{2} \left(s + 1\right) \left(s + 2\right)} \ \Psi(s) = \frac{\left(\begin{matrix}s \left(s + 1\right) \left(s + 2\right) & s \left(s + 1\right) \left(s + 2\right) & s + 1 & s \left(s + 1\right)\\2 s \left(s + 1\right) \left(s + 2\right) & 2 s \left(s + 2\right) & 2 & 2 s\end{matrix}\right)}{s^{2} \left(s + 1\right) \left(s + 2\right)} \]
\[ W(s)  =  \frac{\left(\begin{matrix}s \left(s + 2\right) + s + 2 & s \left(s + 1\right)\\2 s + 4 & 2 s\end{matrix}\right)}{s^{2} \left(s + 1\right) \left(s + 2\right)} = \left(\begin{matrix}\frac{1}{s^{2}} & \frac{1}{s \left(s + 2\right)}\\\frac{2}{s^{2} \left(s + 1\right)} & \frac{2}{s \left(s + 1\right) \left(s + 2\right)}\end{matrix}\right)  \] 
Valore $ s \left(s + 2\right) + s + 2 $ della matrice delle funzioni di trasferimento:
\[ W(s) = \frac{\left(s + 1\right) \left(s + 2\right)}{s^{2} \left(s + 1\right) \left(s + 2\right)} \]\includegraphics[scale = 0.5]{figures/bode_5497868.png}


Valore $ s \left(s + 1\right) $ della matrice delle funzioni di trasferimento:
\[ W(s) = \frac{s \left(s + 1\right)}{s^{2} \left(s + 1\right) \left(s + 2\right)} \]\includegraphics[scale = 0.5]{figures/bode_1201006.png}


Valore $ 2 s + 4 $ della matrice delle funzioni di trasferimento:
\[ W(s) = \frac{2 \left(s + 2\right)}{s^{2} \left(s + 1\right) \left(s + 2\right)} \]\includegraphics[scale = 0.5]{figures/bode_5828221.png}


Valore $ 2 s $ della matrice delle funzioni di trasferimento:
\[ W(s) = \frac{2 s}{s^{2} \left(s + 1\right) \left(s + 2\right)} \]\includegraphics[scale = 0.5]{figures/bode_6729155.png}


\subsubsection{Vediamo le risposte:} 
\section{Esercizio 88 }
 Studiare il sistema \[S:\begin{cases}\overset{\cdot}{x} = \left(\begin{matrix}1 & -3 & -2\\2 & -4 & -2\\-1 & 1 & 0\end{matrix}\right) x+ \left(\begin{matrix}2\\1\\0\end{matrix}\right)u\\y = \left(\begin{matrix}1 & -2 & -1\end{matrix}\right) x +\left(\begin{matrix}0\end{matrix}\right) u\end{cases}\]\subsection{Studio Risposta Libera}
Si studi la risposta libera di un sistema che ha le seguenti caratteristiche: \[A = \left(\begin{matrix}1 & -3 & -2\\2 & -4 & -2\\-1 & 1 & 0\end{matrix}\right)\]
Il determinante di $A-\lambda$ è $ - \lambda^{3} - 3 \lambda^{2} - 2 \lambda $.

Gli autovalori reali sono $\lambda_i = [-2, -1, 0]$.

Gli autovettori associati ai reali sono $ u_i: [  -2: \left(\begin{matrix}1\\1\\0\end{matrix}\right)-1: \left(\begin{matrix}1\\0\\1\end{matrix}\right)0: \left(\begin{matrix}-1\\-1\\1\end{matrix}\right) ]$
.

Da cui posso ricavare le matrici \[U=T^{-1} = \left(\begin{matrix}1 & 1 & -1\\1 & 0 & -1\\0 & 1 & 1\end{matrix}\right), V = T = \left(\begin{matrix}-1 & 2 & 1\\1 & -1 & 0\\-1 & 1 & 1\end{matrix}\right)\]
Che mi trasformano la matrice in \[ D = TAT^{-1} = \left(\begin{matrix}-2 & 0 & 0\\0 & -1 & 0\\0 & 0 & 0\end{matrix}\right) \]
Da cui posso ricavare: \[ \Phi(t) = e^{At} = T^{-1} e^{Dt} T =  T^{-1} \left(\begin{matrix}e^{- 2 t} & 0 & 0\\0 & e^{- t} & 0\\0 & 0 & 1\end{matrix}\right) T\]

\[ = \left(\begin{matrix}1 + e^{- t} - e^{- 2 t} & -1 - e^{- t} + 2 e^{- 2 t} & -1 + e^{- 2 t}\\1 - e^{- 2 t} & -1 + 2 e^{- 2 t} & -1 + e^{- 2 t}\\-1 + e^{- t} & 1 - e^{- t} & 1\end{matrix}\right) \]L'evoluzione libera  a partire dallo stato $ 0 $ è \[ x_L = (\sum_{i=0}^{i=n} e^{\lambda_i t}u_i v_i)x_0=\Phi(t) \cdot x_0 = \left(\begin{matrix}0 & 0 & 0\\0 & 0 & 0\\0 & 0 & 0\end{matrix}\right) \]
La risposta libera è \[ y_L = \Psi(t) x_0 = \left(\begin{matrix}0 & 0 & 0\end{matrix}\right) \]
L'evoluzione libera  a partire dallo stato $ 0 $ è \[ x_L = (\sum_{i=0}^{i=n} e^{\lambda_i t}u_i v_i)x_0=\Phi(t) \cdot x_0 = \left(\begin{matrix}0 & 0 & 0\\0 & 0 & 0\\0 & 0 & 0\end{matrix}\right) \]
La risposta libera è \[ y_L = \Psi(t) x_0 = \left(\begin{matrix}0 & 0 & 0\end{matrix}\right) \]
L'evoluzione libera  a partire dallo stato $ 0 $ è \[ x_L = (\sum_{i=0}^{i=n} e^{\lambda_i t}u_i v_i)x_0=\Phi(t) \cdot x_0 = \left(\begin{matrix}0 & 0 & 0\\0 & 0 & 0\\0 & 0 & 0\end{matrix}\right) \]
La risposta libera è \[ y_L = \Psi(t) x_0 = \left(\begin{matrix}0 & 0 & 0\end{matrix}\right) \]
\subsubsection{Osservabilità}
 I modi naturali osservabili sono quelli tali che 
\[ C \cdot u_i   \neq 0\]
\subsubsection{Eccitabilità}
 I modi naturali eccitabili sono quelli tali che 
\[v_i' \cdot B \neq 0\]

\subsection{Studio Osservabilità}

Studiamone l’osservabilità. Calcoliamo allora O e troviamo I = ker(O):
\[
 O = \begin{pmatrix}C \\ CA \\ CA^2 \end{pmatrix} = \left(\begin{matrix}1 & -2 & -1\\-2 & 4 & 2\\4 & -8 & -4\end{matrix}\right), |O| = 0 \]

O ha rango $ 1 $ quindi il suo nucleo ha dimensione $ 2 $.

Calcolando trovo \[ 
I = ker(O) = \left[ \left(\begin{matrix}2\\1\\0\end{matrix}\right), \  \left(\begin{matrix}1\\0\\1\end{matrix}\right)\right]\]

E $T^{-1}$ viene \[ 
T^{-1} = \left(\begin{matrix}2 & 1 & 0\\1 & 0 & 0\\0 & 1 & 1\end{matrix}\right) \]
\[ 
T = \left(\begin{matrix}0 & 1 & 0\\1 & -2 & 0\\-1 & 2 & 1\end{matrix}\right) \]Calcoliamo allora le matrici del sistema, e vedremo che risultano partizionate come avevamo previsto:
\[ 
\overset{\sim}{A} = T A  T^{-1} = \left(\begin{matrix}0 & 1 & 0\\1 & -2 & 0\\-1 & 2 & 1\end{matrix}\right)\left(\begin{matrix}1 & -3 & -2\\2 & -4 & -2\\-1 & 1 & 0\end{matrix}\right)\left(\begin{matrix}2 & 1 & 0\\1 & 0 & 0\\0 & 1 & 1\end{matrix}\right) = \left(\begin{matrix}0 & 0 & -2\\-1 & -1 & 2\\0 & 0 & -2\end{matrix}\right) \]
\[ 
\overset{\sim}{C} = CT^{-1} = \left(\begin{matrix}1 & -2 & -1\end{matrix}\right)\left(\begin{matrix}2 & 1 & 0\\1 & 0 & 0\\0 & 1 & 1\end{matrix}\right) = \left(\begin{matrix}0 & 0 & -1\end{matrix}\right) = ( 0\ \ \overset{\sim}{C}_2) \]
Con le matrici \[ A_{11} = \left(\begin{matrix}0 & 0\\-1 & -1\end{matrix}\right) , A_{22} = \left(\begin{matrix}-2\end{matrix}\right) \]Quindi infine mi viene che gli autovalori osservabili sono $ -2\  $ e gli inosservabili sono $ 0\ -1\  $.

\subsection{Studio Raggiungibilità}
Ora per studiare la raggiungibilità degli stati calcolo $R = (B\ AB\ ...\ A^{n-1}B)$: \[ R = \left(\begin{matrix}2 & -1 & 1\\1 & 0 & 0\\0 & -1 & 1\end{matrix}\right), |R| = 0 \] 

Vediamo che $rango(R) = 2$ e quindi : \[ \mathfrak{R} = \left[ \left(\begin{matrix}2\\1\\0\end{matrix}\right), \  \left(\begin{matrix}-1\\0\\-1\end{matrix}\right)\right] \]

Tutti gli stati che hanno questa struttura sono, allora, raggiungibili. Mettiamo in evidenza questa struttura;
cambiamo base, e vorremmo avere lo stato espresso come $z = Tx$ tale che, se x è uno stato raggiungibile, allora: \[ z_R = T x_R = \begin{pmatrix} \star  \\ 0 \\0\end{pmatrix}\]

E $T^{-1}$ viene \[ T^{-1} = \left(\begin{matrix}2 & -1 & 0\\1 & 0 & 0\\0 & -1 & 1\end{matrix}\right) \Longrightarrow T = \left(\begin{matrix}0 & 1 & 0\\-1 & 2 & 0\\-1 & 2 & 1\end{matrix}\right) \]
\[ \overset{\sim}{A} = T A  T^{-1} = \left(\begin{matrix}0 & 1 & 0\\-1 & 2 & 0\\-1 & 2 & 1\end{matrix}\right)\left(\begin{matrix}1 & -3 & -2\\2 & -4 & -2\\-1 & 1 & 0\end{matrix}\right)\left(\begin{matrix}2 & -1 & 0\\1 & 0 & 0\\0 & -1 & 1\end{matrix}\right) = \left(\begin{matrix}0 & 0 & -2\\1 & -1 & -2\\0 & 0 & -2\end{matrix}\right) \]Con le matrici \[ \overset{\sim}{A}_{11} = \left(\begin{matrix}0 & 0\\1 & -1\end{matrix}\right) , \overset{\sim}{A}_{22} = \left(\begin{matrix}-2\end{matrix}\right)  \]e le matrici \[ \overset{\sim}{B} = TB = \left(\begin{matrix}1\\0\\0\end{matrix}\right)  \]
Ora ne calcoliamo la raggiungibilità: \[ \overset{\sim}{H}(t) = e^{\overset{\sim}{A}t}\overset{\sim}{B} = \begin{pmatrix} e^{\overset{\sim}{A}_{11}t} &  \star \\ 0 & e^{\overset{\sim}{A}_{22}t} \end{pmatrix} \begin{pmatrix} \overset{\sim}{B}_1 \\ 0 \end{pmatrix} = \begin{pmatrix} e^{\overset{\sim}{A_{11}t}}\overset{\sim}{B_1} \\ 0 \end{pmatrix} \]
Quindi infine mi viene che gli autovalori ragg sono $ 0, -1,  $ e gli irrag sono $ -2,  $
\subsection{Scomposizione di Kalman}
I miei sottospazi di riferimento sono:	\[ \mathfrak{I} = \left[ \left(\begin{matrix}2\\1\\0\end{matrix}\right), \  \left(\begin{matrix}1\\0\\1\end{matrix}\right)\right], \mathfrak{R} = \left[ \left(\begin{matrix}2\\1\\0\end{matrix}\right), \  \left(\begin{matrix}-1\\0\\-1\end{matrix}\right)\right] \]
La matrice dei vettori di base di I e R è \[ \chi_1 =  \left(\begin{matrix}-2 & 0\\-1 & -1\\0 & 2\end{matrix}\right) \]
Ci sono più righe che colonne quindi sicuro l'intersezione c'è.

\paragraph{Per quanto riguarda Chi2:} $ \chi_2 | \chi_2 \oplus \chi_1 = \mathfrak{R} $ è \[ \chi_2 = \left(\begin{matrix}\end{matrix}\right) \]

\paragraph{Per quanto riguarda Chi3:} $ \chi_3 | \chi_3 \oplus \chi_1 = \mathfrak{I} $ è \[ \chi_3 = \left(\begin{matrix}\end{matrix}\right) \]

\paragraph{Per quanto riguarda Chi4:} $ \chi_4 | \chi_1 \oplus \chi_2 \oplus  \chi_3 \oplus \chi_4 = \mathbb{R} $ è \[ \chi_4 = \left(\begin{matrix}0\\0\\1\end{matrix}\right) \]
\paragraph{Ora facciamo T inversa:} \[ T^{-1} = (\chi_1\ \chi_2\ \chi_3\ \chi_4\ ) = \left(\begin{matrix}-2 & 0 & 0\\-1 & -1 & 0\\0 & 2 & 1\end{matrix}\right) \]
e quindi \[T = \left(\begin{matrix}- \frac{1}{2} & 0 & 0\\\frac{1}{2} & -1 & 0\\-1 & 2 & 1\end{matrix}\right)\]
\[ \widetilde{A} = TAT^{-1} = \left(\begin{matrix}- \frac{1}{2} & \frac{3}{2} & 1\\- \frac{3}{2} & \frac{5}{2} & 1\\2 & -4 & -2\end{matrix}\right) * T^{-1} = T*\left(\begin{matrix}1 & -1 & -2\\0 & 0 & -2\\1 & -1 & 0\end{matrix}\right) =\left(\begin{matrix}- \frac{1}{2} & \frac{1}{2} & 1\\\frac{1}{2} & - \frac{1}{2} & 1\\0 & 0 & -2\end{matrix}\right) \]

\[ \widetilde{B} = T B = \left(\begin{matrix}-1\\0\\0\end{matrix}\right) \]

\[ \widetilde{C} = C T^{-1} = \left(\begin{matrix}0 & 0 & -1\end{matrix}\right) \]
\[Phi(t) = \left(\begin{matrix}\frac{1}{2} + \frac{e^{- t}}{2} & \frac{1}{2} - \frac{e^{- t}}{2} & \frac{1}{2} - \frac{e^{- 2 t}}{2}\\\frac{1}{2} - \frac{e^{- t}}{2} & \frac{1}{2} + \frac{e^{- t}}{2} & \frac{1}{2} - \frac{e^{- 2 t}}{2}\\0 & 0 & e^{- 2 t}\end{matrix}\right) \]

\subsection{Studio Funzione di trasferimento}

\[ (sI-A) = \left(\begin{matrix}s - 1 & 3 & 2\\-2 & s + 4 & 2\\1 & -1 & s\end{matrix}\right), |sI-A| = s \left(s + 1\right) \left(s + 2\right) \]
\[ \Phi(s) = (sI-A)^{-1} = \frac{\left(\begin{matrix}s^{2} + 4 s + 2 & - 3 s - 2 & - 2 s - 2\\2 s + 2 & \left(s - 2\right) \left(s + 1\right) & - 2 s - 2\\- s - 2 & s + 2 & \left(s + 1\right) \left(s + 2\right)\end{matrix}\right)}{s \left(s + 1\right) \left(s + 2\right)} \]

Le funzioni caratteristiche sono \[\begin{array}{rcl}  H(s) & = & \Phi(s)B \\ \Psi(s) & = & C \Phi(s)\\ W(s) & = & C(sI-A)^{-1}B  \end{array} \]

e quindi \[ H(s)  =  \frac{\left(\begin{matrix}2 s^{2} + 5 s + 2\\s^{2} + 3 s + 2\\- s - 2\end{matrix}\right)}{s \left(s + 1\right) \left(s + 2\right)} \ \Psi(s) = \frac{\left(\begin{matrix}s \left(s + 1\right) & 2 s \left(- s - 1\right) & s \left(- s - 1\right)\end{matrix}\right)}{s \left(s + 1\right) \left(s + 2\right)} \]
\[ W(s)  =  \frac{\left(\begin{matrix}0\end{matrix}\right)}{s \left(s + 1\right) \left(s + 2\right)} = \left(\begin{matrix}0\end{matrix}\right)  \] 
Il grafico di bode è:
Il numeratore della funzione è zero quindi niente graficiNo Nyquist
\subsubsection{Vediamo le risposte:}
\[ y_L = \Psi(s) x_0 = \frac{\left(\begin{matrix}s \left(s + 1\right) & 2 s \left(- s - 1\right) & s \left(- s - 1\right)\end{matrix}\right)}{s \left(s + 1\right) \left(s + 2\right)} 0 = \frac{\left(\begin{matrix}0 & 0 & 0\end{matrix}\right)}{s \left(s + 1\right) \left(s + 2\right)} \]

\[ y_L = \Psi(s) x_0 = \frac{\left(\begin{matrix}s \left(s + 1\right) & 2 s \left(- s - 1\right) & s \left(- s - 1\right)\end{matrix}\right)}{s \left(s + 1\right) \left(s + 2\right)} 0 = \frac{\left(\begin{matrix}0 & 0 & 0\end{matrix}\right)}{s \left(s + 1\right) \left(s + 2\right)} \]

\[ y_L = \Psi(s) x_0 = \frac{\left(\begin{matrix}s \left(s + 1\right) & 2 s \left(- s - 1\right) & s \left(- s - 1\right)\end{matrix}\right)}{s \left(s + 1\right) \left(s + 2\right)} 0 = \frac{\left(\begin{matrix}0 & 0 & 0\end{matrix}\right)}{s \left(s + 1\right) \left(s + 2\right)} \]
 
\section{Esercizio 99 }
 Studiare il sistema \[S:\begin{cases}\overset{\cdot}{x} = \left(\begin{matrix}2 & 1 & 0\\0 & 2 & 1\\0 & 0 & 2\end{matrix}\right) x+ \left(\begin{matrix}1\\1\\0\end{matrix}\right)u\\y = \left(\begin{matrix}1 & -1 & 0\end{matrix}\right) x +\left(\begin{matrix}0\end{matrix}\right) u\end{cases}\]\subsection{Studio Risposta Libera}
Si studi la risposta libera di un sistema che ha le seguenti caratteristiche: \[A = \left(\begin{matrix}2 & 1 & 0\\0 & 2 & 1\\0 & 0 & 2\end{matrix}\right)\]
Il determinante di $A-\lambda$ è $ - \lambda^{3} + 6 \lambda^{2} - 12 \lambda + 8 $.

Gli autovalori reali sono $\lambda_i = [2]$.

Gli autovettori associati ai reali sono $ u_i: [  2: (\left(\begin{matrix}1\\0\\0\end{matrix}\right)) ]$
.
La matrice $A$ non è diagonalizzabile, quindi devo fare Jordan.

Da cui posso ricavare le matrici \[U=T^{-1} = \left(\begin{matrix}1 & 0 & 0\\0 & 1 & 0\\0 & 0 & 1\end{matrix}\right), V = T = \left(\begin{matrix}1 & 0 & 0\\0 & 1 & 0\\0 & 0 & 1\end{matrix}\right)\]
Che mi trasformano la matrice in \[ D = TAT^{-1} = \left(\begin{matrix}2 & 1 & 0\\0 & 2 & 1\\0 & 0 & 2\end{matrix}\right) \]
Da cui posso ricavare: \[ \Phi(t) = e^{At} = T^{-1} e^{Dt} T =  T^{-1} \left(\begin{matrix}e^{2 t} & t e^{2 t} & \frac{t^{2} e^{2 t}}{2}\\0 & e^{2 t} & t e^{2 t}\\0 & 0 & e^{2 t}\end{matrix}\right) T\]

\[ = \left(\begin{matrix}e^{2 t} & t e^{2 t} & \frac{t^{2} e^{2 t}}{2}\\0 & e^{2 t} & t e^{2 t}\\0 & 0 & e^{2 t}\end{matrix}\right) \]L'evoluzione libera  a partire dallo stato $ 0 $ è \[ x_L = (\sum_{i=0}^{i=n} e^{\lambda_i t}u_i v_i)x_0=\Phi(t) \cdot x_0 = \left(\begin{matrix}0 & 0 & 0\\0 & 0 & 0\\0 & 0 & 0\end{matrix}\right) \]
La risposta libera è \[ y_L = \Psi(t) x_0 = \left(\begin{matrix}0 & 0 & 0\end{matrix}\right) \]
L'evoluzione libera  a partire dallo stato $ 0 $ è \[ x_L = (\sum_{i=0}^{i=n} e^{\lambda_i t}u_i v_i)x_0=\Phi(t) \cdot x_0 = \left(\begin{matrix}0 & 0 & 0\\0 & 0 & 0\\0 & 0 & 0\end{matrix}\right) \]
La risposta libera è \[ y_L = \Psi(t) x_0 = \left(\begin{matrix}0 & 0 & 0\end{matrix}\right) \]
L'evoluzione libera  a partire dallo stato $ 0 $ è \[ x_L = (\sum_{i=0}^{i=n} e^{\lambda_i t}u_i v_i)x_0=\Phi(t) \cdot x_0 = \left(\begin{matrix}0 & 0 & 0\\0 & 0 & 0\\0 & 0 & 0\end{matrix}\right) \]
La risposta libera è \[ y_L = \Psi(t) x_0 = \left(\begin{matrix}0 & 0 & 0\end{matrix}\right) \]
\subsubsection{Osservabilità}
 I modi naturali osservabili sono quelli tali che 
\[ C \cdot u_i   \neq 0\]
\subsubsection{Eccitabilità}
 I modi naturali eccitabili sono quelli tali che 
\[v_i' \cdot B \neq 0\]

\subsection{Studio Osservabilità}

Studiamone l’osservabilità. Calcoliamo allora O e troviamo I = ker(O):
\[
 O = \begin{pmatrix}C \\ CA \\ CA^2 \end{pmatrix} = \left(\begin{matrix}1 & -1 & 0\\2 & -1 & -1\\4 & 0 & -3\end{matrix}\right), |O| = 1 \]
Eccezione 1: O ha rango pieno quindi finisco qui.

\subsection{Studio Raggiungibilità}
Ora per studiare la raggiungibilità degli stati calcolo $R = (B\ AB\ ...\ A^{n-1}B)$: \[ R = \left(\begin{matrix}1 & 3 & 8\\1 & 2 & 4\\0 & 0 & 0\end{matrix}\right), |R| = 0 \] 

Vediamo che $rango(R) = 2$ e quindi : \[ \mathfrak{R} = \left[ \left(\begin{matrix}1\\1\\0\end{matrix}\right), \  \left(\begin{matrix}3\\2\\0\end{matrix}\right)\right] \]

Tutti gli stati che hanno questa struttura sono, allora, raggiungibili. Mettiamo in evidenza questa struttura;
cambiamo base, e vorremmo avere lo stato espresso come $z = Tx$ tale che, se x è uno stato raggiungibile, allora: \[ z_R = T x_R = \begin{pmatrix} \star  \\ 0 \\0\end{pmatrix}\]

E $T^{-1}$ viene \[ T^{-1} = \left(\begin{matrix}1 & 3 & 0\\1 & 2 & 0\\0 & 0 & 1\end{matrix}\right) \Longrightarrow T = \left(\begin{matrix}-2 & 3 & 0\\1 & -1 & 0\\0 & 0 & 1\end{matrix}\right) \]
\[ \overset{\sim}{A} = T A  T^{-1} = \left(\begin{matrix}-2 & 3 & 0\\1 & -1 & 0\\0 & 0 & 1\end{matrix}\right)\left(\begin{matrix}2 & 1 & 0\\0 & 2 & 1\\0 & 0 & 2\end{matrix}\right)\left(\begin{matrix}1 & 3 & 0\\1 & 2 & 0\\0 & 0 & 1\end{matrix}\right) = \left(\begin{matrix}0 & -4 & 3\\1 & 4 & -1\\0 & 0 & 2\end{matrix}\right) \]Con le matrici \[ \overset{\sim}{A}_{11} = \left(\begin{matrix}0 & -4\\1 & 4\end{matrix}\right) , \overset{\sim}{A}_{22} = \left(\begin{matrix}2\end{matrix}\right)  \]e le matrici \[ \overset{\sim}{B} = TB = \left(\begin{matrix}1\\0\\0\end{matrix}\right)  \]
Ora ne calcoliamo la raggiungibilità: \[ \overset{\sim}{H}(t) = e^{\overset{\sim}{A}t}\overset{\sim}{B} = \begin{pmatrix} e^{\overset{\sim}{A}_{11}t} &  \star \\ 0 & e^{\overset{\sim}{A}_{22}t} \end{pmatrix} \begin{pmatrix} \overset{\sim}{B}_1 \\ 0 \end{pmatrix} = \begin{pmatrix} e^{\overset{\sim}{A_{11}t}}\overset{\sim}{B_1} \\ 0 \end{pmatrix} \]
Quindi infine mi viene che gli autovalori ragg sono $ 2,  $ e gli irrag sono $ 2,  $
\subsection{Scomposizione di Kalman}
I miei sottospazi di riferimento sono:	\[ \mathfrak{I} = \left[ \right], \mathfrak{R} = \left[ \left(\begin{matrix}1\\1\\0\end{matrix}\right), \  \left(\begin{matrix}3\\2\\0\end{matrix}\right)\right] \]
La matrice dei vettori di base di I e R è \[ \chi_1 =  \left(\begin{matrix}\end{matrix}\right) \]Ed il determinante è ininfluente.
\paragraph{Per quanto riguarda Chi2:} $ \chi_2 | \chi_2 \oplus \chi_1 = \mathfrak{R} $ è \[ \chi_2 = \left(\begin{matrix}1 & 3\\1 & 2\\0 & 0\end{matrix}\right) \]

\paragraph{Per quanto riguarda Chi3:} $ \chi_3 | \chi_3 \oplus \chi_1 = \mathfrak{I} $ è \[ \chi_3 = \left(\begin{matrix}\end{matrix}\right) \]

\paragraph{Per quanto riguarda Chi4:} $ \chi_4 | \chi_1 \oplus \chi_2 \oplus  \chi_3 \oplus \chi_4 = \mathbb{R} $ è \[ \chi_4 = \left(\begin{matrix}0\\0\\1\end{matrix}\right) \]
\paragraph{Ora facciamo T inversa:} \[ T^{-1} = (\chi_1\ \chi_2\ \chi_3\ \chi_4\ ) = \left(\begin{matrix}1 & 3 & 0\\1 & 2 & 0\\0 & 0 & 1\end{matrix}\right) \]
e quindi \[T = \left(\begin{matrix}-2 & 3 & 0\\1 & -1 & 0\\0 & 0 & 1\end{matrix}\right)\]
\[ \widetilde{A} = TAT^{-1} = \left(\begin{matrix}-4 & 4 & 3\\2 & -1 & -1\\0 & 0 & 2\end{matrix}\right) * T^{-1} = T*\left(\begin{matrix}3 & 8 & 0\\2 & 4 & 1\\0 & 0 & 2\end{matrix}\right) =\left(\begin{matrix}0 & -4 & 3\\1 & 4 & -1\\0 & 0 & 2\end{matrix}\right) \]

\[ \widetilde{B} = T B = \left(\begin{matrix}1\\0\\0\end{matrix}\right) \]

\[ \widetilde{C} = C T^{-1} = \left(\begin{matrix}0 & 1 & 0\end{matrix}\right) \]
\[Phi(t) = \left(\begin{matrix}- 2 e^{2 t} + \frac{- 2 t^{2} e^{2 t} + 3 t e^{2 t}}{t} & - 6 e^{2 t} + \frac{2 \left(- 2 t^{2} e^{2 t} + 3 t e^{2 t}\right)}{t} & - t^{2} e^{2 t} + 3 t e^{2 t}\\e^{2 t} + \frac{t^{2} e^{2 t} - t e^{2 t}}{t} & 3 e^{2 t} + \frac{2 \left(t^{2} e^{2 t} - t e^{2 t}\right)}{t} & \frac{t^{2} e^{2 t}}{2} - t e^{2 t}\\0 & 0 & e^{2 t}\end{matrix}\right) \]

\subsection{Studio Funzione di trasferimento}

\[ (sI-A) = \left(\begin{matrix}s - 2 & -1 & 0\\0 & s - 2 & -1\\0 & 0 & s - 2\end{matrix}\right), |sI-A| = \left(s - 2\right)^{3} \]
\[ \Phi(s) = (sI-A)^{-1} = \frac{\left(\begin{matrix}\left(s - 2\right)^{2} & s - 2 & 1\\0 & \left(s - 2\right)^{2} & s - 2\\0 & 0 & \left(s - 2\right)^{2}\end{matrix}\right)}{\left(s - 2\right)^{3}} \]

Le funzioni caratteristiche sono \[\begin{array}{rcl}  H(s) & = & \Phi(s)B \\ \Psi(s) & = & C \Phi(s)\\ W(s) & = & C(sI-A)^{-1}B  \end{array} \]

e quindi \[ H(s)  =  \frac{\left(\begin{matrix}s + \left(s - 2\right)^{2} - 2\\\left(s - 2\right)^{2}\\0\end{matrix}\right)}{\left(s - 2\right)^{3}} \ \Psi(s) = \frac{\left(\begin{matrix}\left(s - 2\right)^{2} & s - \left(s - 2\right)^{2} - 2 & 3 - s\end{matrix}\right)}{\left(s - 2\right)^{3}} \]
\[ W(s)  =  \frac{\left(\begin{matrix}s - 2\end{matrix}\right)}{\left(s - 2\right)^{3}} = \left(\begin{matrix}\frac{1}{\left(s - 2\right)^{2}}\end{matrix}\right)  \] 
Il grafico di bode è:
\[ W(s) = \frac{s - 2}{\left(s - 2\right)^{3}} \]\includegraphics[scale = 0.5]{figures/bode_6241981.png}


Il grafico di Nyquist è:
\includegraphics[scale = 0.5]{figures/nyquist_7461042.png}nel tempo continuo è \[ t e^{2 t} \theta\left(t\right) \]
\subsubsection{Vediamo le risposte:}
\[ y_L = \Psi(s) x_0 = \frac{\left(\begin{matrix}\left(s - 2\right)^{2} & s - \left(s - 2\right)^{2} - 2 & 3 - s\end{matrix}\right)}{\left(s - 2\right)^{3}} 0 = \frac{\left(\begin{matrix}0 & 0 & 0\end{matrix}\right)}{\left(s - 2\right)^{3}} \]

\[ y_L = \Psi(s) x_0 = \frac{\left(\begin{matrix}\left(s - 2\right)^{2} & s - \left(s - 2\right)^{2} - 2 & 3 - s\end{matrix}\right)}{\left(s - 2\right)^{3}} 0 = \frac{\left(\begin{matrix}0 & 0 & 0\end{matrix}\right)}{\left(s - 2\right)^{3}} \]

\[ y_L = \Psi(s) x_0 = \frac{\left(\begin{matrix}\left(s - 2\right)^{2} & s - \left(s - 2\right)^{2} - 2 & 3 - s\end{matrix}\right)}{\left(s - 2\right)^{3}} 0 = \frac{\left(\begin{matrix}0 & 0 & 0\end{matrix}\right)}{\left(s - 2\right)^{3}} \]
 
\section{Esercizio 111 }
 Studiare il sistema \[S:\begin{cases}\overset{\cdot}{x} = \left(\begin{matrix}0 & 1\\0 & -1\end{matrix}\right) x+ \left(\begin{matrix}0\\1\end{matrix}\right)u\\y = \left(\begin{matrix}1 & 1\\2 & 0\end{matrix}\right) x +\left(\begin{matrix}0\\0\end{matrix}\right) u\end{cases}\]\subsection{Studio Risposta Libera}
Si studi la risposta libera di un sistema che ha le seguenti caratteristiche: \[A = \left(\begin{matrix}0 & 1\\0 & -1\end{matrix}\right)\]
Il determinante di $A-\lambda$ è $ \lambda^{2} + \lambda $.

Gli autovalori reali sono $\lambda_i = [-1, 0]$.

Gli autovettori associati ai reali sono $ u_i: [  -1: \left(\begin{matrix}-1\\1\end{matrix}\right)0: \left(\begin{matrix}1\\0\end{matrix}\right) ]$
.

Da cui posso ricavare le matrici \[U=T^{-1} = \left(\begin{matrix}-1 & 1\\1 & 0\end{matrix}\right), V = T = \left(\begin{matrix}0 & 1\\1 & 1\end{matrix}\right)\]
Che mi trasformano la matrice in \[ D = TAT^{-1} = \left(\begin{matrix}-1 & 0\\0 & 0\end{matrix}\right) \]
Da cui posso ricavare: \[ \Phi(t) = e^{At} = T^{-1} e^{Dt} T =  T^{-1} \left(\begin{matrix}e^{- t} & 0\\0 & 1\end{matrix}\right) T\]

\[ = \left(\begin{matrix}1 & 1 - e^{- t}\\0 & e^{- t}\end{matrix}\right) \]\subsubsection{Osservabilità}
 I modi naturali osservabili sono quelli tali che 
\[ C \cdot u_i   \neq 0\]
\subsubsection{Eccitabilità}
 I modi naturali eccitabili sono quelli tali che 
\[v_i' \cdot B \neq 0\]

\subsection{Studio Osservabilità}

Studiamone l’osservabilità. Calcoliamo allora O e troviamo I = ker(O):
\[
 O = \begin{pmatrix}C \\ CA \\ CA^2 \end{pmatrix} = \left(\begin{matrix}1 & 1\\2 & 0\\0 & 0\\0 & 2\end{matrix}\right), |O| = 999 \]

O ha rango $ 2 $ quindi il suo nucleo ha dimensione $ 2 $.

Calcolando trovo \[ 
I = ker(O) = \left[ \right]\]
2O ha rango pieno quindi finisco qui.

\subsection{Studio Raggiungibilità}
Ora per studiare la raggiungibilità degli stati calcolo $R = (B\ AB\ ...\ A^{n-1}B)$: \[ R = \left(\begin{matrix}0 & 1\\1 & -1\end{matrix}\right), |R| = -1 \] 

Vediamo che $rango(R) = 2$ e quindi : \[ \mathfrak{R} = \left[ \left(\begin{matrix}0\\1\end{matrix}\right), \  \left(\begin{matrix}1\\-1\end{matrix}\right)\right] \]

Tutti gli stati che hanno questa struttura sono, allora, raggiungibili. Mettiamo in evidenza questa struttura;
cambiamo base, e vorremmo avere lo stato espresso come $z = Tx$ tale che, se x è uno stato raggiungibile, allora: \[ z_R = T x_R = \begin{pmatrix} \star  \\ 0 \\0\end{pmatrix}\]

E $T^{-1}$ viene \[ T^{-1} = \left(\begin{matrix}0 & 1\\1 & -1\end{matrix}\right) \Longrightarrow T = \left(\begin{matrix}1 & 1\\1 & 0\end{matrix}\right) \]
\[ \overset{\sim}{A} = T A  T^{-1} = \left(\begin{matrix}1 & 1\\1 & 0\end{matrix}\right)\left(\begin{matrix}0 & 1\\0 & -1\end{matrix}\right)\left(\begin{matrix}0 & 1\\1 & -1\end{matrix}\right) = \left(\begin{matrix}0 & 0\\1 & -1\end{matrix}\right) \]Con le matrici \[ \overset{\sim}{A}_{11} = \left(\begin{matrix}0 & 0\\1 & -1\end{matrix}\right) , \overset{\sim}{A}_{22} = \left(\begin{matrix}\end{matrix}\right)  \]e le matrici \[ \overset{\sim}{B} = TB = \left(\begin{matrix}1\\0\end{matrix}\right)  \]
Ora ne calcoliamo la raggiungibilità: \[ \overset{\sim}{H}(t) = e^{\overset{\sim}{A}t}\overset{\sim}{B} = \begin{pmatrix} e^{\overset{\sim}{A}_{11}t} &  \star \\ 0 & e^{\overset{\sim}{A}_{22}t} \end{pmatrix} \begin{pmatrix} \overset{\sim}{B}_1 \\ 0 \end{pmatrix} = \begin{pmatrix} e^{\overset{\sim}{A_{11}t}}\overset{\sim}{B_1} \\ 0 \end{pmatrix} \]
Quindi infine mi viene che gli autovalori ragg sono $ 0, -1,  $ e gli irrag sono $  $
\subsection{Scomposizione di Kalman}
I miei sottospazi di riferimento sono:	\[ \mathfrak{I} = \left[ \right], \mathfrak{R} = \left[ \left(\begin{matrix}0\\1\end{matrix}\right), \  \left(\begin{matrix}1\\-1\end{matrix}\right)\right] \]
La matrice dei vettori di base di I e R è \[ \chi_1 =  \left(\begin{matrix}\end{matrix}\right) \]Ed il determinante è ininfluente.
\paragraph{Per quanto riguarda Chi2:} $ \chi_2 | \chi_2 \oplus \chi_1 = \mathfrak{R} $ è \[ \chi_2 = \left(\begin{matrix}0 & 1\\1 & -1\end{matrix}\right) \]

\paragraph{Per quanto riguarda Chi3:} $ \chi_3 | \chi_3 \oplus \chi_1 = \mathfrak{I} $ è \[ \chi_3 = \left(\begin{matrix}\end{matrix}\right) \]

\paragraph{Per quanto riguarda Chi4:} $ \chi_4 | \chi_1 \oplus \chi_2 \oplus  \chi_3 \oplus \chi_4 = \mathbb{R} $ è \[ \chi_4 = \left(\begin{matrix}\end{matrix}\right) \]
\paragraph{Ora facciamo T inversa:} \[ T^{-1} = (\chi_1\ \chi_2\ \chi_3\ \chi_4\ ) = \left(\begin{matrix}0 & 1\\1 & -1\end{matrix}\right) \]
e quindi \[T = \left(\begin{matrix}1 & 1\\1 & 0\end{matrix}\right)\]
\[ \widetilde{A} = TAT^{-1} = \left(\begin{matrix}0 & 0\\0 & 1\end{matrix}\right) * T^{-1} = T*\left(\begin{matrix}1 & -1\\-1 & 1\end{matrix}\right) =\left(\begin{matrix}0 & 0\\1 & -1\end{matrix}\right) \]

\[ \widetilde{B} = T B = \left(\begin{matrix}1\\0\end{matrix}\right) \]

\[ \widetilde{C} = C T^{-1} = \left(\begin{matrix}1 & 0\\0 & 2\end{matrix}\right) \]
\[Phi(t) = \left(\begin{matrix}1 & 0\\1 - e^{- t} & e^{- t}\end{matrix}\right) \]

\subsection{Studio Funzione di trasferimento}

\[ (sI-A) = \left(\begin{matrix}s & -1\\0 & s + 1\end{matrix}\right), |sI-A| = s \left(s + 1\right) \]
\[ \Phi(s) = (sI-A)^{-1} = \frac{\left(\begin{matrix}s + 1 & 1\\0 & s\end{matrix}\right)}{s \left(s + 1\right)} \]

Le funzioni caratteristiche sono \[\begin{array}{rcl}  H(s) & = & \Phi(s)B \\ \Psi(s) & = & C \Phi(s)\\ W(s) & = & C(sI-A)^{-1}B  \end{array} \]

e quindi \[ H(s)  =  \frac{\left(\begin{matrix}1\\s\end{matrix}\right)}{s \left(s + 1\right)} \ \Psi(s) = \frac{\left(\begin{matrix}s + 1 & s + 1\\2 s + 2 & 2\end{matrix}\right)}{s \left(s + 1\right)} \]
\[ W(s)  =  \frac{\left(\begin{matrix}s + 1\\2\end{matrix}\right)}{s \left(s + 1\right)} = \left(\begin{matrix}\frac{1}{s}\\\frac{2}{s \left(s + 1\right)}\end{matrix}\right)  \] 
Valore $ s + 1 $ della matrice delle funzioni di trasferimento:
\[ W(s) = \frac{s + 1}{s \left(s + 1\right)} \]\includegraphics[scale = 0.5]{figures/bode_2306467.png}


Valore $ 2 $ della matrice delle funzioni di trasferimento:
\[ W(s) = \frac{2}{s \left(s + 1\right)} \]\includegraphics[scale = 0.5]{figures/bode_4731690.png}


Valore $ 0 $ della matrice delle funzioni di trasferimento:
\includegraphics[scale = 0.5]{figures/nyquist_6301942.png}nel tempo continuo è \[ \theta\left(t\right) \]
Valore $ 1 $ della matrice delle funzioni di trasferimento:
\includegraphics[scale = 0.5]{figures/nyquist_6695394.png}nel tempo continuo è \[ 2 \left(e^{t} - 1\right) e^{- t} \theta\left(t\right) \]
\subsubsection{Vediamo le risposte:} 
\section{Esercizio 122 }
 Studiare il sistema \[S:\begin{cases}\overset{\cdot}{x} = \left(\begin{matrix}0 & 0\\1 & -2\end{matrix}\right) x+ \left(\begin{matrix}2 & 0\\1 & 1\end{matrix}\right)u\\y = \left(\begin{matrix}0 & 1\end{matrix}\right) x +\left(\begin{matrix}0 & 0\end{matrix}\right) u\end{cases}\]\subsection{Studio Risposta Libera}
Si studi la risposta libera di un sistema che ha le seguenti caratteristiche: \[A = \left(\begin{matrix}0 & 0\\1 & -2\end{matrix}\right)\]
Il determinante di $A-\lambda$ è $ \lambda^{2} + 2 \lambda $.

Gli autovalori reali sono $\lambda_i = [-2, 0]$.

Gli autovettori associati ai reali sono $ u_i: [  -2: \left(\begin{matrix}0\\1\end{matrix}\right)0: \left(\begin{matrix}2\\1\end{matrix}\right) ]$
.

Da cui posso ricavare le matrici \[U=T^{-1} = \left(\begin{matrix}0 & 2\\1 & 1\end{matrix}\right), V = T = \left(\begin{matrix}- \frac{1}{2} & 1\\\frac{1}{2} & 0\end{matrix}\right)\]
Che mi trasformano la matrice in \[ D = TAT^{-1} = \left(\begin{matrix}-2 & 0\\0 & 0\end{matrix}\right) \]
Da cui posso ricavare: \[ \Phi(t) = e^{At} = T^{-1} e^{Dt} T =  T^{-1} \left(\begin{matrix}e^{- 2 t} & 0\\0 & 1\end{matrix}\right) T\]

\[ = \left(\begin{matrix}1 & 0\\\frac{1}{2} - \frac{e^{- 2 t}}{2} & e^{- 2 t}\end{matrix}\right) \]\subsubsection{Osservabilità}
 I modi naturali osservabili sono quelli tali che 
\[ C \cdot u_i   \neq 0\]
\subsubsection{Eccitabilità}
 I modi naturali eccitabili sono quelli tali che 
\[v_i' \cdot B \neq 0\]

\subsection{Studio Osservabilità}

Studiamone l’osservabilità. Calcoliamo allora O e troviamo I = ker(O):
\[
 O = \begin{pmatrix}C \\ CA \\ CA^2 \end{pmatrix} = \left(\begin{matrix}0 & 1\\1 & -2\end{matrix}\right), |O| = -1 \]
Eccezione 1: O ha rango pieno quindi finisco qui.

\subsection{Studio Raggiungibilità}
Ora per studiare la raggiungibilità degli stati calcolo $R = (B\ AB\ ...\ A^{n-1}B)$: \[ R = \left(\begin{matrix}2 & 0 & 0 & 0\\1 & 1 & 0 & -2\end{matrix}\right), |R| = 999 \] 

Vediamo che $rango(R) = 2$ e quindi : \[ \mathfrak{R} = \left[ \left(\begin{matrix}2\\1\end{matrix}\right), \  \left(\begin{matrix}0\\1\end{matrix}\right)\right] \]

Tutti gli stati che hanno questa struttura sono, allora, raggiungibili. Mettiamo in evidenza questa struttura;
cambiamo base, e vorremmo avere lo stato espresso come $z = Tx$ tale che, se x è uno stato raggiungibile, allora: \[ z_R = T x_R = \begin{pmatrix} \star  \\ 0 \\0\end{pmatrix}\]

E $T^{-1}$ viene \[ T^{-1} = \left(\begin{matrix}2 & 0\\1 & 1\end{matrix}\right) \Longrightarrow T = \left(\begin{matrix}\frac{1}{2} & 0\\- \frac{1}{2} & 1\end{matrix}\right) \]
\[ \overset{\sim}{A} = T A  T^{-1} = \left(\begin{matrix}\frac{1}{2} & 0\\- \frac{1}{2} & 1\end{matrix}\right)\left(\begin{matrix}0 & 0\\1 & -2\end{matrix}\right)\left(\begin{matrix}2 & 0\\1 & 1\end{matrix}\right) = \left(\begin{matrix}0 & 0\\0 & -2\end{matrix}\right) \]Con le matrici \[ \overset{\sim}{A}_{11} = \left(\begin{matrix}0 & 0\\0 & -2\end{matrix}\right) , \overset{\sim}{A}_{22} = \left(\begin{matrix}\end{matrix}\right)  \]e le matrici \[ \overset{\sim}{B} = TB = \left(\begin{matrix}1 & 0\\0 & 1\end{matrix}\right)  \]
Ora ne calcoliamo la raggiungibilità: \[ \overset{\sim}{H}(t) = e^{\overset{\sim}{A}t}\overset{\sim}{B} = \begin{pmatrix} e^{\overset{\sim}{A}_{11}t} &  \star \\ 0 & e^{\overset{\sim}{A}_{22}t} \end{pmatrix} \begin{pmatrix} \overset{\sim}{B}_1 \\ 0 \end{pmatrix} = \begin{pmatrix} e^{\overset{\sim}{A_{11}t}}\overset{\sim}{B_1} \\ 0 \end{pmatrix} \]
Quindi infine mi viene che gli autovalori ragg sono $ 0, -2,  $ e gli irrag sono $  $
\subsection{Scomposizione di Kalman}
I miei sottospazi di riferimento sono:	\[ \mathfrak{I} = \left[ \right], \mathfrak{R} = \left[ \left(\begin{matrix}2\\1\end{matrix}\right), \  \left(\begin{matrix}0\\1\end{matrix}\right)\right] \]
La matrice dei vettori di base di I e R è \[ \chi_1 =  \left(\begin{matrix}\end{matrix}\right) \]Ed il determinante è ininfluente.
\paragraph{Per quanto riguarda Chi2:} $ \chi_2 | \chi_2 \oplus \chi_1 = \mathfrak{R} $ è \[ \chi_2 = \left(\begin{matrix}2 & 0\\1 & 1\end{matrix}\right) \]

\paragraph{Per quanto riguarda Chi3:} $ \chi_3 | \chi_3 \oplus \chi_1 = \mathfrak{I} $ è \[ \chi_3 = \left(\begin{matrix}\end{matrix}\right) \]

\paragraph{Per quanto riguarda Chi4:} $ \chi_4 | \chi_1 \oplus \chi_2 \oplus  \chi_3 \oplus \chi_4 = \mathbb{R} $ è \[ \chi_4 = \left(\begin{matrix}\end{matrix}\right) \]
\paragraph{Ora facciamo T inversa:} \[ T^{-1} = (\chi_1\ \chi_2\ \chi_3\ \chi_4\ ) = \left(\begin{matrix}2 & 0\\1 & 1\end{matrix}\right) \]
e quindi \[T = \left(\begin{matrix}\frac{1}{2} & 0\\- \frac{1}{2} & 1\end{matrix}\right)\]
\[ \widetilde{A} = TAT^{-1} = \left(\begin{matrix}0 & 0\\1 & -2\end{matrix}\right) * T^{-1} = T*\left(\begin{matrix}0 & 0\\0 & -2\end{matrix}\right) =\left(\begin{matrix}0 & 0\\0 & -2\end{matrix}\right) \]

\[ \widetilde{B} = T B = \left(\begin{matrix}1 & 0\\0 & 1\end{matrix}\right) \]

\[ \widetilde{C} = C T^{-1} = \left(\begin{matrix}1 & 1\end{matrix}\right) \]
\[Phi(t) = \left(\begin{matrix}1 & 0\\0 & e^{- 2 t}\end{matrix}\right) \]

\subsection{Studio Funzione di trasferimento}

\[ (sI-A) = \left(\begin{matrix}s & 0\\-1 & s + 2\end{matrix}\right), |sI-A| = s \left(s + 2\right) \]
\[ \Phi(s) = (sI-A)^{-1} = \frac{\left(\begin{matrix}s + 2 & 0\\1 & s\end{matrix}\right)}{s \left(s + 2\right)} \]

Le funzioni caratteristiche sono \[\begin{array}{rcl}  H(s) & = & \Phi(s)B \\ \Psi(s) & = & C \Phi(s)\\ W(s) & = & C(sI-A)^{-1}B  \end{array} \]

e quindi \[ H(s)  =  \frac{\left(\begin{matrix}2 s + 4 & 0\\s + 2 & s\end{matrix}\right)}{s \left(s + 2\right)} \ \Psi(s) = \frac{\left(\begin{matrix}1 & s\end{matrix}\right)}{s \left(s + 2\right)} \]
\[ W(s)  =  \frac{\left(\begin{matrix}s + 2 & s\end{matrix}\right)}{s \left(s + 2\right)} = \left(\begin{matrix}\frac{1}{s} & \frac{1}{s + 2}\end{matrix}\right)  \] 
Valore $ s + 2 $ della matrice delle funzioni di trasferimento:
\[ W(s) = \frac{s + 2}{s \left(s + 2\right)} \]\includegraphics[scale = 0.5]{figures/bode_4788265.png}


Valore $ s $ della matrice delle funzioni di trasferimento:
\[ W(s) = \frac{s}{s \left(s + 2\right)} \]\includegraphics[scale = 0.5]{figures/bode_5037644.png}


\subsubsection{Vediamo le risposte:} 
\section{Esercizio 133 }
 Studiare il sistema \[S:\begin{cases}\overset{\cdot}{x} = \left(\begin{matrix}-10\end{matrix}\right) x+ \left(\begin{matrix}1\end{matrix}\right)u\\y = \left(\begin{matrix}1\end{matrix}\right) x +\left(\begin{matrix}0\end{matrix}\right) u\end{cases}\]\subsection{Studio Risposta Libera}
Si studi la risposta libera di un sistema che ha le seguenti caratteristiche: \[A = \left(\begin{matrix}-10\end{matrix}\right)\]
Il determinante di $A-\lambda$ è $ - \lambda - 10 $.

Gli autovalori reali sono $\lambda_i = [-10]$.

Gli autovettori associati ai reali sono $ u_i: [  -10: \left(\begin{matrix}1\end{matrix}\right) ]$
.

Da cui posso ricavare le matrici \[U=T^{-1} = \left(\begin{matrix}1\end{matrix}\right), V = T = \left(\begin{matrix}1\end{matrix}\right)\]
Che mi trasformano la matrice in \[ D = TAT^{-1} = \left(\begin{matrix}-10\end{matrix}\right) \]
Da cui posso ricavare: \[ \Phi(t) = e^{At} = T^{-1} e^{Dt} T =  T^{-1} \left(\begin{matrix}e^{- 10 t}\end{matrix}\right) T\]

\[ = \left(\begin{matrix}e^{- 10 t}\end{matrix}\right) \]L'evoluzione libera  a partire dallo stato $ 0 $ è \[ x_L = (\sum_{i=0}^{i=n} e^{\lambda_i t}u_i v_i)x_0=\Phi(t) \cdot x_0 = \left(\begin{matrix}0\end{matrix}\right) \]
La risposta libera è \[ y_L = \Psi(t) x_0 = \left(\begin{matrix}0\end{matrix}\right) \]
\subsubsection{Osservabilità}
 I modi naturali osservabili sono quelli tali che 
\[ C \cdot u_i   \neq 0\]
\subsubsection{Eccitabilità}
 I modi naturali eccitabili sono quelli tali che 
\[v_i' \cdot B \neq 0\]

\subsection{Studio Funzione di trasferimento}

\[ (sI-A) = \left(\begin{matrix}s + 10\end{matrix}\right), |sI-A| = s + 10 \]
\[ \Phi(s) = (sI-A)^{-1} = \frac{\left(\begin{matrix}1\end{matrix}\right)}{s + 10} \]

Le funzioni caratteristiche sono \[\begin{array}{rcl}  H(s) & = & \Phi(s)B \\ \Psi(s) & = & C \Phi(s)\\ W(s) & = & C(sI-A)^{-1}B  \end{array} \]

e quindi \[ H(s)  =  \frac{\left(\begin{matrix}1\end{matrix}\right)}{s + 10} \ \Psi(s) = \frac{\left(\begin{matrix}1\end{matrix}\right)}{s + 10} \]
\[ W(s)  =  \frac{\left(\begin{matrix}1\end{matrix}\right)}{s + 10} = \left(\begin{matrix}\frac{1}{s + 10}\end{matrix}\right)  \] 
Il grafico di bode è:
\[ W(s) = \frac{1}{s + 10} \]\includegraphics[scale = 0.5]{figures/bode_8363299.png}


Il grafico di Nyquist è:
\includegraphics[scale = 0.5]{figures/nyquist_2857279.png}nel tempo continuo è \[ e^{- 10 t} \theta\left(t\right) \]
\subsubsection{Vediamo le risposte:}
\[ y_L = \Psi(s) x_0 = \frac{\left(\begin{matrix}1\end{matrix}\right)}{s + 10} 0 = \frac{\left(\begin{matrix}0\end{matrix}\right)}{s + 10} \]
 
\section{Esercizio 144 }
 Studiare il sistema \[S:\begin{cases}\overset{\cdot}{x} = \left(\begin{matrix}0 & 1 & 0\\1 & 0 & 0\\0 & 0 & -2\end{matrix}\right) x+ \left(\begin{matrix}1\\0\\1\end{matrix}\right)u\\y = \left(\begin{matrix}1 & -1 & 0\end{matrix}\right) x +\left(\begin{matrix}0\end{matrix}\right) u\end{cases}\]\subsection{Studio Risposta Libera}
Si studi la risposta libera di un sistema che ha le seguenti caratteristiche: \[A = \left(\begin{matrix}0 & 1 & 0\\1 & 0 & 0\\0 & 0 & -2\end{matrix}\right)\]
Il determinante di $A-\lambda$ è $ - \lambda^{3} - 2 \lambda^{2} + \lambda + 2 $.

Gli autovalori reali sono $\lambda_i = [-2, -1, 1]$.

Gli autovettori associati ai reali sono $ u_i: [  -2: \left(\begin{matrix}0\\0\\1\end{matrix}\right)-1: \left(\begin{matrix}-1\\1\\0\end{matrix}\right)1: \left(\begin{matrix}1\\1\\0\end{matrix}\right) ]$
.

Da cui posso ricavare le matrici \[U=T^{-1} = \left(\begin{matrix}0 & -1 & 1\\0 & 1 & 1\\1 & 0 & 0\end{matrix}\right), V = T = \left(\begin{matrix}0 & 0 & 1\\- \frac{1}{2} & \frac{1}{2} & 0\\\frac{1}{2} & \frac{1}{2} & 0\end{matrix}\right)\]
Che mi trasformano la matrice in \[ D = TAT^{-1} = \left(\begin{matrix}-2 & 0 & 0\\0 & -1 & 0\\0 & 0 & 1\end{matrix}\right) \]
Da cui posso ricavare: \[ \Phi(t) = e^{At} = T^{-1} e^{Dt} T =  T^{-1} \left(\begin{matrix}e^{- 2 t} & 0 & 0\\0 & e^{- t} & 0\\0 & 0 & e^{t}\end{matrix}\right) T\]

\[ = \left(\begin{matrix}\frac{e^{t}}{2} + \frac{e^{- t}}{2} & \frac{e^{t}}{2} - \frac{e^{- t}}{2} & 0\\\frac{e^{t}}{2} - \frac{e^{- t}}{2} & \frac{e^{t}}{2} + \frac{e^{- t}}{2} & 0\\0 & 0 & e^{- 2 t}\end{matrix}\right) \]\subsubsection{Osservabilità}
 I modi naturali osservabili sono quelli tali che 
\[ C \cdot u_i   \neq 0\]
\subsubsection{Eccitabilità}
 I modi naturali eccitabili sono quelli tali che 
\[v_i' \cdot B \neq 0\]

\subsection{Studio Osservabilità}

Studiamone l’osservabilità. Calcoliamo allora O e troviamo I = ker(O):
\[
 O = \begin{pmatrix}C \\ CA \\ CA^2 \end{pmatrix} = \left(\begin{matrix}1 & -1 & 0\\-1 & 1 & 0\\1 & -1 & 0\end{matrix}\right), |O| = 0 \]

O ha rango $ 1 $ quindi il suo nucleo ha dimensione $ 2 $.

Calcolando trovo \[ 
I = ker(O) = \left[ \left(\begin{matrix}1\\1\\0\end{matrix}\right), \  \left(\begin{matrix}0\\0\\1\end{matrix}\right)\right]\]

E $T^{-1}$ viene \[ 
T^{-1} = \left(\begin{matrix}1 & 0 & 1\\1 & 0 & 0\\0 & 1 & 0\end{matrix}\right) \]
\[ 
T = \left(\begin{matrix}0 & 1 & 0\\0 & 0 & 1\\1 & -1 & 0\end{matrix}\right) \]Calcoliamo allora le matrici del sistema, e vedremo che risultano partizionate come avevamo previsto:
\[ 
\overset{\sim}{A} = T A  T^{-1} = \left(\begin{matrix}0 & 1 & 0\\0 & 0 & 1\\1 & -1 & 0\end{matrix}\right)\left(\begin{matrix}0 & 1 & 0\\1 & 0 & 0\\0 & 0 & -2\end{matrix}\right)\left(\begin{matrix}1 & 0 & 1\\1 & 0 & 0\\0 & 1 & 0\end{matrix}\right) = \left(\begin{matrix}1 & 0 & 1\\0 & -2 & 0\\0 & 0 & -1\end{matrix}\right) \]
\[ 
\overset{\sim}{C} = CT^{-1} = \left(\begin{matrix}1 & -1 & 0\end{matrix}\right)\left(\begin{matrix}1 & 0 & 1\\1 & 0 & 0\\0 & 1 & 0\end{matrix}\right) = \left(\begin{matrix}0 & 0 & 1\end{matrix}\right) = ( 0\ \ \overset{\sim}{C}_2) \]
Con le matrici \[ A_{11} = \left(\begin{matrix}1 & 0\\0 & -2\end{matrix}\right) , A_{22} = \left(\begin{matrix}-1\end{matrix}\right) \]Quindi infine mi viene che gli autovalori osservabili sono $ -1\  $ e gli inosservabili sono $ 1\ -2\  $.

\subsection{Studio Raggiungibilità}
Ora per studiare la raggiungibilità degli stati calcolo $R = (B\ AB\ ...\ A^{n-1}B)$: \[ R = \left(\begin{matrix}1 & 0 & 1\\0 & 1 & 0\\1 & -2 & 4\end{matrix}\right), |R| = 3 \] 

Vediamo che $rango(R) = 3$ e quindi : \[ \mathfrak{R} = \left[ \left(\begin{matrix}1\\0\\1\end{matrix}\right), \  \left(\begin{matrix}0\\1\\-2\end{matrix}\right), \  \left(\begin{matrix}1\\0\\4\end{matrix}\right)\right] \]

Tutti gli stati che hanno questa struttura sono, allora, raggiungibili. Mettiamo in evidenza questa struttura;
cambiamo base, e vorremmo avere lo stato espresso come $z = Tx$ tale che, se x è uno stato raggiungibile, allora: \[ z_R = T x_R = \begin{pmatrix} \star  \\ 0 \\0\end{pmatrix}\]

E $T^{-1}$ viene \[ T^{-1} = \left(\begin{matrix}1 & 0 & 1\\0 & 1 & 0\\1 & -2 & 4\end{matrix}\right) \Longrightarrow T = \left(\begin{matrix}\frac{4}{3} & - \frac{2}{3} & - \frac{1}{3}\\0 & 1 & 0\\- \frac{1}{3} & \frac{2}{3} & \frac{1}{3}\end{matrix}\right) \]
\[ \overset{\sim}{A} = T A  T^{-1} = \left(\begin{matrix}\frac{4}{3} & - \frac{2}{3} & - \frac{1}{3}\\0 & 1 & 0\\- \frac{1}{3} & \frac{2}{3} & \frac{1}{3}\end{matrix}\right)\left(\begin{matrix}0 & 1 & 0\\1 & 0 & 0\\0 & 0 & -2\end{matrix}\right)\left(\begin{matrix}1 & 0 & 1\\0 & 1 & 0\\1 & -2 & 4\end{matrix}\right) = \left(\begin{matrix}0 & 0 & 2\\1 & 0 & 1\\0 & 1 & -2\end{matrix}\right) \]Con le matrici \[ \overset{\sim}{A}_{11} = \left(\begin{matrix}0 & 0 & 2\\1 & 0 & 1\\0 & 1 & -2\end{matrix}\right) , \overset{\sim}{A}_{22} = \left(\begin{matrix}\end{matrix}\right)  \]e le matrici \[ \overset{\sim}{B} = TB = \left(\begin{matrix}1\\0\\0\end{matrix}\right)  \]
Ora ne calcoliamo la raggiungibilità: \[ \overset{\sim}{H}(t) = e^{\overset{\sim}{A}t}\overset{\sim}{B} = \begin{pmatrix} e^{\overset{\sim}{A}_{11}t} &  \star \\ 0 & e^{\overset{\sim}{A}_{22}t} \end{pmatrix} \begin{pmatrix} \overset{\sim}{B}_1 \\ 0 \end{pmatrix} = \begin{pmatrix} e^{\overset{\sim}{A_{11}t}}\overset{\sim}{B_1} \\ 0 \end{pmatrix} \]
Quindi infine mi viene che gli autovalori ragg sono $ 1, -1, -2,  $ e gli irrag sono $  $
\subsection{Scomposizione di Kalman}
I miei sottospazi di riferimento sono:	\[ \mathfrak{I} = \left[ \left(\begin{matrix}1\\1\\0\end{matrix}\right), \  \left(\begin{matrix}0\\0\\1\end{matrix}\right)\right], \mathfrak{R} = \left[ \left(\begin{matrix}1\\0\\1\end{matrix}\right), \  \left(\begin{matrix}0\\1\\-2\end{matrix}\right), \  \left(\begin{matrix}1\\0\\4\end{matrix}\right)\right] \]
La matrice dei vettori di base di I e R è \[ \chi_1 =  \left(\begin{matrix}1 & 0\\1 & 0\\-1 & 1\end{matrix}\right) \]
Ci sono più righe che colonne quindi sicuro l'intersezione c'è.

\paragraph{Per quanto riguarda Chi2:} $ \chi_2 | \chi_2 \oplus \chi_1 = \mathfrak{R} $ è \[ \chi_2 = \left(\begin{matrix}1\\0\\1\end{matrix}\right) \]

\paragraph{Per quanto riguarda Chi3:} $ \chi_3 | \chi_3 \oplus \chi_1 = \mathfrak{I} $ è \[ \chi_3 = \left(\begin{matrix}\end{matrix}\right) \]

\paragraph{Per quanto riguarda Chi4:} $ \chi_4 | \chi_1 \oplus \chi_2 \oplus  \chi_3 \oplus \chi_4 = \mathbb{R} $ è \[ \chi_4 = \left(\begin{matrix}\end{matrix}\right) \]
\paragraph{Ora facciamo T inversa:} \[ T^{-1} = (\chi_1\ \chi_2\ \chi_3\ \chi_4\ ) = \left(\begin{matrix}1 & 0 & 1\\1 & 0 & 0\\-1 & 1 & 1\end{matrix}\right) \]
e quindi \[T = \left(\begin{matrix}0 & 1 & 0\\-1 & 2 & 1\\1 & -1 & 0\end{matrix}\right)\]
\[ \widetilde{A} = TAT^{-1} = \left(\begin{matrix}1 & 0 & 0\\2 & -1 & -2\\-1 & 1 & 0\end{matrix}\right) * T^{-1} = T*\left(\begin{matrix}1 & 0 & 0\\1 & 0 & 1\\2 & -2 & -2\end{matrix}\right) =\left(\begin{matrix}1 & 0 & 1\\3 & -2 & 0\\0 & 0 & -1\end{matrix}\right) \]

\[ \widetilde{B} = T B = \left(\begin{matrix}0\\0\\1\end{matrix}\right) \]

\[ \widetilde{C} = C T^{-1} = \left(\begin{matrix}0 & 0 & 1\end{matrix}\right) \]
\[Phi(t) = \left(\begin{matrix}e^{t} & 0 & \frac{e^{t}}{2} - \frac{e^{- t}}{2}\\e^{t} - e^{- 2 t} & e^{- 2 t} & \frac{e^{t}}{2} - \frac{3 e^{- t}}{2} + e^{- 2 t}\\0 & 0 & e^{- t}\end{matrix}\right) \]

\subsection{Studio Funzione di trasferimento}

\[ (sI-A) = \left(\begin{matrix}s & -1 & 0\\-1 & s & 0\\0 & 0 & s + 2\end{matrix}\right), |sI-A| = \left(s - 1\right) \left(s + 1\right) \left(s + 2\right) \]
\[ \Phi(s) = (sI-A)^{-1} = \frac{\left(\begin{matrix}s \left(s + 2\right) & s + 2 & 0\\s + 2 & s \left(s + 2\right) & 0\\0 & 0 & s^{2} - 1\end{matrix}\right)}{\left(s - 1\right) \left(s + 1\right) \left(s + 2\right)} \]

Le funzioni caratteristiche sono \[\begin{array}{rcl}  H(s) & = & \Phi(s)B \\ \Psi(s) & = & C \Phi(s)\\ W(s) & = & C(sI-A)^{-1}B  \end{array} \]

e quindi \[ H(s)  =  \frac{\left(\begin{matrix}s \left(s + 2\right)\\s + 2\\s^{2} - 1\end{matrix}\right)}{\left(s - 1\right) \left(s + 1\right) \left(s + 2\right)} \ \Psi(s) = \frac{\left(\begin{matrix}s^{2} + s - 2 & - s^{2} - s + 2 & 0\end{matrix}\right)}{\left(s - 1\right) \left(s + 1\right) \left(s + 2\right)} \]
\[ W(s)  =  \frac{\left(\begin{matrix}s^{2} + s - 2\end{matrix}\right)}{\left(s - 1\right) \left(s + 1\right) \left(s + 2\right)} = \left(\begin{matrix}\frac{1}{s + 1}\end{matrix}\right)  \] 
Il grafico di bode è:
\[ W(s) = \frac{\left(s - 1\right) \left(s + 2\right)}{\left(s - 1\right) \left(s + 1\right) \left(s + 2\right)} \]\includegraphics[scale = 0.5]{figures/bode_1557046.png}


Il grafico di Nyquist è:
\includegraphics[scale = 0.5]{figures/nyquist_6777377.png}nel tempo continuo è \[ e^{- t} \theta\left(t\right) \]
\subsubsection{Vediamo le risposte:} 




























\end{document}
