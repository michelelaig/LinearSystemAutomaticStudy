\documentclass{article}
\usepackage{amsmath}
\usepackage{amsfonts}
\usepackage[a4paper,width=140mm,top=15mm,bottom=15mm]{geometry}
\usepackage{hyperref}
\usepackage{mathtools}
\usepackage{graphicx}
\graphicspath{ {./figures/} }

\hypersetup{
    colorlinks,
    citecolor=black,
    filecolor=black,
    linkcolor=black,
    urlcolor=black
}

\DeclareUnicodeCharacter{2212}{-}




\title{Esercizi}
\author{Michele Leigheb}
\date{}
\begin{document}
\maketitle
\tableofcontents{}
\section{Complessi}
\begin{itemize}
	\item \(\displaystyle 2^{(a+ib)} = 2^a (\cos(b \ln(2)) + i\sin(b \ln(2))) \)
	\item \(\displaystyle 3^{(a+ib)} = 3^a (\cos(b \ln(3)) + i\sin(b \ln(3))) \)
	\item \(\displaystyle e^{(a+ib)} = e^a (\cos(b)) + i\sin(b) \)
	\item \(\displaystyle \alpha^{(a+ib)} = e^{\alpha} (\cos(b)\ln(\alpha)) + i\sin(b\ln(\alpha)) \)
\end{itemize}



\section{Esercizi}

\section{Esercizio 157 }
 Studiare il sistema \[S:\begin{cases}\overset{\cdot}{x} = \left(\begin{matrix}-1 & 0 & 0 & 0\\0 & 0 & 0 & 0\\0 & 0 & 0 & 1\\0 & 0 & 0 & 0\end{matrix}\right) x+ \left(\begin{matrix}0 & 0\\1 & 0\\0 & 0\\0 & 1\end{matrix}\right)u\\y = \left(\begin{matrix}0 & 1 & 1 & 0\\0 & 0 & 1 & 0\end{matrix}\right) x +\left(\begin{matrix}0 & 0\\1 & 0\end{matrix}\right) u\end{cases}\]\subsection{Studio Risposta Libera}
Si studi la risposta libera di un sistema che ha le seguenti caratteristiche: \[A = \left(\begin{matrix}-1 & 0 & 0 & 0\\0 & 0 & 0 & 0\\0 & 0 & 0 & 1\\0 & 0 & 0 & 0\end{matrix}\right)\]
Il determinante di $A-\lambda$ è $ - \lambda^{3} \left(- \lambda - 1\right) $.

Gli autovalori reali sono $\lambda_i = [-1, 0]$.

Gli autovettori associati ai reali sono $ u_i: [  -1: \left(\begin{matrix}1\\0\\0\\0\end{matrix}\right)0: (\left(\begin{matrix}0\\1\\0\\0\end{matrix}\right), \left(\begin{matrix}0\\0\\1\\0\end{matrix}\right)) ]$
.
La matrice $A$ non è diagonalizzabile, quindi devo fare Jordan.

Da cui posso ricavare le matrici \[U=T^{-1} = \left(\begin{matrix}1 & 0 & 0 & 0\\0 & 0 & 0 & 1\\0 & 1 & 0 & 0\\0 & 0 & 1 & 0\end{matrix}\right), V = T = \left(\begin{matrix}1 & 0 & 0 & 0\\0 & 0 & 1 & 0\\0 & 0 & 0 & 1\\0 & 1 & 0 & 0\end{matrix}\right)\]
Che mi trasformano la matrice in \[ D = TAT^{-1} = \left(\begin{matrix}-1 & 0 & 0 & 0\\0 & 0 & 1 & 0\\0 & 0 & 0 & 0\\0 & 0 & 0 & 0\end{matrix}\right) \]
Da cui posso ricavare: \[ \Phi(t) = e^{At} = T^{-1} e^{Dt} T =  T^{-1} \left(\begin{matrix}e^{- t} & 0 & 0 & 0\\0 & 1 & t & 0\\0 & 0 & 1 & 0\\0 & 0 & 0 & 1\end{matrix}\right) T\]

\[ = \left(\begin{matrix}e^{- t} & 0 & 0 & 0\\0 & 1 & 0 & 0\\0 & 0 & 1 & t\\0 & 0 & 0 & 1\end{matrix}\right) \]\[ \Psi(t) = \left(\begin{matrix}0 & 1 & 1 & t\\0 & 0 & 1 & t\end{matrix}\right), H(t) =  \left(\begin{matrix}0 & 0\\1 & 0\\0 & t\\0 & 1\end{matrix}\right),W(t) = \left(\begin{matrix}1 & t\\1 & t\end{matrix}\right)\]\subsubsection{Osservabilità}
 I modi naturali osservabili sono quelli tali che 
\[ C \cdot u_i   \neq 0\]
\subsubsection{Eccitabilità}
 I modi naturali eccitabili sono quelli tali che 
\[v_i' \cdot B \neq 0\]

\subsection{Studio Funzione di trasferimento}

\[ (sI-A) = \left(\begin{matrix}s + 1 & 0 & 0 & 0\\0 & s & 0 & 0\\0 & 0 & s & -1\\0 & 0 & 0 & s\end{matrix}\right), |sI-A| = s^{3} \left(s + 1\right) \]
\[ \Phi(s) = (sI-A)^{-1} = \frac{\left(\begin{matrix}s^{3} & 0 & 0 & 0\\0 & s^{2} \left(s + 1\right) & 0 & 0\\0 & 0 & s^{2} \left(s + 1\right) & s \left(s + 1\right)\\0 & 0 & 0 & s^{2} \left(s + 1\right)\end{matrix}\right)}{s^{3} \left(s + 1\right)} \]

Le funzioni caratteristiche sono \[\begin{array}{rcl}  H(s) & = & \Phi(s)B \\ \Psi(s) & = & C \Phi(s)\\ W(s) & = & C(sI-A)^{-1}B  \end{array} \]
\[ H(s) = \left(\begin{matrix}s^{3} & 0 & 0 & 0\\0 & s^{2} \left(s + 1\right) & 0 & 0\\0 & 0 & s^{2} \left(s + 1\right) & s \left(s + 1\right)\\0 & 0 & 0 & s^{2} \left(s + 1\right)\end{matrix}\right)\cdot \frac{\left(\begin{matrix}0 & 0\\1 & 0\\0 & 0\\0 & 1\end{matrix}\right)}{s^{3} \left(s + 1\right)} = \frac{\left(\begin{matrix}0 & 0\\s^{2} \left(s + 1\right) & 0\\0 & s \left(s + 1\right)\\0 & s^{2} \left(s + 1\right)\end{matrix}\right)}{s^{3} \left(s + 1\right)} \]\[ \Psi(s) = \frac{\left(\begin{matrix}0 & 1 & 1 & 0\\0 & 0 & 1 & 0\end{matrix}\right)}{s^{3} \left(s + 1\right)}\cdot\left(\begin{matrix}s^{3} & 0 & 0 & 0\\0 & s^{2} \left(s + 1\right) & 0 & 0\\0 & 0 & s^{2} \left(s + 1\right) & s \left(s + 1\right)\\0 & 0 & 0 & s^{2} \left(s + 1\right)\end{matrix}\right)  = \frac{\left(\begin{matrix}0 & s^{2} \left(s + 1\right) & s^{2} \left(s + 1\right) & s \left(s + 1\right)\\0 & 0 & s^{2} \left(s + 1\right) & s \left(s + 1\right)\end{matrix}\right)}{s^{3} \left(s + 1\right)} \]
e quindi \[ H(s)  =  \frac{\left(\begin{matrix}0 & 0\\s^{2} \left(s + 1\right) & 0\\0 & s \left(s + 1\right)\\0 & s^{2} \left(s + 1\right)\end{matrix}\right)}{s^{3} \left(s + 1\right)} \ \Psi(s) = \frac{\left(\begin{matrix}0 & s^{2} \left(s + 1\right) & s^{2} \left(s + 1\right) & s \left(s + 1\right)\\0 & 0 & s^{2} \left(s + 1\right) & s \left(s + 1\right)\end{matrix}\right)}{s^{3} \left(s + 1\right)} \]
\[ W(s)  =  \frac{\left(\begin{matrix}s^{2} \left(s + 1\right) & s \left(s + 1\right)\\s^{3} \left(s + 1\right) & s \left(s + 1\right)\end{matrix}\right)}{s^{3} \left(s + 1\right)} = \left(\begin{matrix}\frac{1}{s} & \frac{1}{s^{2}}\\1 & \frac{1}{s^{2}}\end{matrix}\right)  \] 
Valore $ s^{2} \left(s + 1\right) $ della matrice delle funzioni di trasferimento:
\[ W(s) = \frac{s^{2} \left(s + 1\right)}{s^{3} \left(s + 1\right)} \]\includegraphics[scale = 0.5]{figures/bode_7875577.png}


Valore $ s \left(s + 1\right) $ della matrice delle funzioni di trasferimento:
\[ W(s) = \frac{s \left(s + 1\right)}{s^{3} \left(s + 1\right)} \]\includegraphics[scale = 0.5]{figures/bode_8131886.png}


Valore $ s^{3} \left(s + 1\right) $ della matrice delle funzioni di trasferimento:
\[ W(s) = \frac{s^{3} \left(s + 1\right)}{s^{3} \left(s + 1\right)} \]\includegraphics[scale = 0.5]{figures/bode_2818649.png}


Valore $ s \left(s + 1\right) $ della matrice delle funzioni di trasferimento:
\[ W(s) = \frac{s \left(s + 1\right)}{s^{3} \left(s + 1\right)} \]\includegraphics[scale = 0.5]{figures/bode_1236601.png}


\subsubsection{Vediamo le risposte:} 




























\end{document}
